\documentclass[a4paper,12pt]{report} %%%{article}

\usepackage{cmap} % searchable PDFs
\usepackage[T2A]{fontenc} % scalable fonts
\usepackage[utf8]{inputenc} % input in UTF8
\usepackage[english,russian]{babel} % dashes on linebreaks
\usepackage{indentfirst} % indents in paragraphs
\usepackage{amstext,amssymb,amsfonts,amsmath,mathtext,enumerate,float}
\usepackage[left=25mm,right=2cm,top=2cm,bottom=2cm,bindingoffset=0cm]{geometry}
\usepackage[unicode]{hyperref}
\usepackage{graphicx}
\usepackage{ulem} % strikethrough
\usepackage{verbatim} % multiline comments
\usepackage{hhline}

\usepackage{listings}
\usepackage{color}
 
\lstset{extendedchars=\true}
  
\definecolor{dkgreen}{rgb}{0,0.6,0}
\definecolor{gray}{rgb}{0.5,0.5,0.5}
\definecolor{mauve}{rgb}{0.58,0,0.82}
 
\lstset{ %
  columns=flexible,
%  language=C,                     % the language of the code
  basicstyle=\footnotesize\ttfamily,       % the size of the fonts that are used for the code
  numbers=left,                   % where to put the line-numbers
  numberstyle=\tiny\color{gray},  % the style that is used for the line-numbers
  stepnumber=1,                   % the step between two line-numbers. If it's 1, each line will be numbered
  numbersep=5pt,                  % how far the line-numbers are from the code
  backgroundcolor=\color{white},  % choose the background color. You must add \usepackage{color}
  showspaces=false,               % show spaces adding particular underscores
  showstringspaces=false,         % underline spaces within strings
  showtabs=false,                 % show tabs within strings adding particular underscores
%  frame=single,                   % adds a frame around the code
  rulecolor=\color{black},        % if not set, the frame-color may be changed on line-breaks within not-black text (e.g. comments (green here))
  tabsize=4,                      % sets default tabsize
  captionpos=b,                   % sets the caption-position to bottom
  breaklines=true,                % sets automatic line breaking
  breakatwhitespace=false,        % sets if automatic breaks should only happen at whitespace
  title=\lstname,                 % show the filename of files included with \lstinputlisting; also try caption instead of title
  keywordstyle=\color{blue},      % keyword style
  commentstyle=\color{dkgreen},   % comment style
  stringstyle=\color{mauve},      % string literal style
%  escapeinside={\%*}{*)},         % if you want to add LaTeX within your code
  mathescape=true,
  morekeywords={*,...},           % if you want to add more keywords to the set
  deletekeywords={...},           % if you want to delete keywords from the given language
  aboveskip=1em,
  belowskip=-1em
}


\usepackage{mathtools}
\newcommand{\defeq}{\vcentcolon=}
\newcommand{\eqdef}{=\vcentcolon}

\renewcommand{\contentsname}{Содержание} 
\setcounter{secnumdepth}{0}
\setcounter{tocdepth}{3}

% \linespread{1.3}
\sloppy % align=justify


\begin{document}

\section{Основные определения и аксиоматика}

Мы рассматриваем процесс компиляции как функцию, получающую на вход набор файлов, содержащие исходный код, и некоторый контекст. Контекст представляется в виде множества порождений. Возвращаемое значение функции, то есть результат компиляции~--- это тоже некоторое множество порождений. Здесь мы понимаем порождение не как класс-файл, а как некоторый неделимый атрибут, например, имя класса, поле класса, метод, элементарный тип и т.д. Это связано с тем, что в языках, подобных Java, наблюдаются зависимости разного рода; например, использование поля~--- совсем не то же самое, что вызов статического метода. При этом различные зависимости по-разному влияют на процесс инкрементальной компиляции при изменениях исходных файлов. К примеру, если разработчик изменил сигнатуру метода, то следует рассмотреть для перекомпиляции все файлы, где вызывался этот метод; однако если было изменено только тело метода, то такая модификация никак не могла затронуть другие файлы. Такой неоднородный граф зависимостей нельзя получить, сравнивая лишь содержимое файлов.

В полученных на вход исходных файлах могут содержаться упоминания-ссылки на такие атрибуты, определённые в других файлах, таким образом, для успешной компиляции соответствующие атрибуты должны быть представлены в контексте; в противном случае функция будет не определена (такую ситуацию назовём ``недоопределённостью''). Также возможен случай, когда в исходном файле определяется тот же самый атрибут, который уже представлен в контексте; в таком случае возникает неоднозначность, и функция будет не определена (такую ситуацию назовём ``переопределённостью''). Наконец, существуют вырожденные входы, функция на которых не определена ни в каком контексте.

Обозначим $\Sigma$~--- множество входов (исходных файлов), $\Omega$~--- множество выходов (порождений); эти множества не более чем счётны. Тогда частичная функция порождения выходов по входам (функция компиляции) будет выглядеть так:
$$gen : 2^\Omega \times 2^\Sigma \to 2^\Omega$$
Если $gen(\omega, \sigma) = \omega^\prime$, то будем говорить, что $\omega^\prime$~--- результат порождения входа $\sigma$ в контексте $\omega$.

Введём разбиение результата $gen(\omega, \sigma) = \omega^\prime$ на классы следующим образом. Класс $B^0_{\omega^\prime}$~--- это те порождения или части порождений, которые зависят только от входа $\sigma$. Класс $B^1_{\omega^\prime}$ - те порождения или части порождений, которые зависят только от входа и порождений класса $B^0_{\omega}$ из контекста. По индукции, класс $B^i_{\omega^\prime}$ - те порождения или части порождений, которые зависят только от входа и порождений классов $B^j_{\omega}$, где $j < i$.\\

Для функции порождения был установлен следующий набор аксиом:

\begin{enumerate}
	\item Аксиома о переопределённости: если $gen(\omega,\sigma)$~--- определена, то $\omega\cap gen(\omega,\sigma) = \varnothing$;
	
	\item Аксиома о дизъюнктном разбиении: если $gen(\omega,\sigma) = \omega^\prime$, то существует единственное дизъюнктное разбиение $\omega^\prime=\bigcup^\varnothing_{s\in\sigma}\omega^\prime_s$, 
	удовлетворяющее свойству 

	$$\forall s\in\sigma : gen(\omega\cup\omega^\prime\setminus\omega^\prime_s,\{s\})=\omega^\prime_s$$

	\item Аксиома об эквивалентных контекстах: $\forall s\in\Sigma,\; \forall\omega,\:\omega^\prime\subseteq\Omega:$ если $s$ невырожденное, $gen(\omega,\{s\})$ определено и $B^0_{\omega} = B^0_{\omega^\prime}$, то и $gen(\omega^\prime,\{s\})$ определено, более того, $gen(\omega,\{s\}) = gen(\omega^\prime,\{s\})$;

	\item Аксиома о минимально необходимом контексте: если $gen(\omega,\{s\})$~--- определена, то в $\omega$ существует наименьшее по включению подмножество $d_\omega(s)$, такое, что $gen(d_\omega(s), \{s\})$~--- определена;

	\item Аксиома о недоопределённости: если $gen(\omega, \sigma) = \omega^\prime$ и для какого-то $s\in\sigma$ существует $s_1\in\sigma$, такой, что $d_{\omega\cup\omega^\prime\setminus\omega^\prime_s}(s) \cap \omega^\prime_{s_1}\ne\varnothing$, то $gen(\omega, \sigma\setminus s_1)$ не определена;
	
	\item Аксиома о сужении контекста: если $gen(\omega,\{s\})$~--- определено, то для произвольного $\omega^\prime\subseteq\omega$, такого, что $d_\omega(s)\subseteq\omega^\prime$, $gen(\omega^\prime, \{s\})$ тоже определено и равно $gen(\omega,\{s\})$.
\end{enumerate}

Как видно, аксиомы довольно естественны в том смысле, что не накладывают на функцию компиляции ограничений, оторванных от реального мира.\\

Приведём очевидное, но полезное следствие из аксиомы о дизъюнктном разбиении:

\textbf{Следствие.}
Если $gen(\omega, \sigma) = \omega^\prime$, то для любого $\tau \subseteq \sigma$ можно определить $\omega^\prime_\tau = \bigcup\limits_{s\in\tau}\omega^\prime_s$, при этом $\tau_1 \cap \tau_2 = \varnothing \Leftrightarrow \omega^\prime_{\tau_1} \cap \omega^\prime_{\tau_2} = \varnothing$.\\

% --------------------------------------------------

Введём понятие дифференциала, которое поможет сформулировать теоремы об инкрементальной компиляции и о переиспользовании порождений.

\textbf{Определение.} Пусть $\omega$, $\tilde{\omega}$ --- множества выходов, $\sigma$ --- множество входов. Известно, что определены $gen(\omega, \sigma) = \omega^\prime$, $gen(\tilde{\omega}, \sigma) = \tilde{\omega}^\prime$. Тогда дифференциал $\partial\dfrac{\omega}{\tilde{\omega}}\sigma = \partial$ --- это наименьшее подмножество $\sigma$, удовлетворяющее свойству: 
$gen(\omega \cup \omega^\prime_{\partial}, \sigma\setminus\partial)$ и
$gen(\tilde{\omega} \cup \tilde{\omega}^\prime_{\partial}, \sigma\setminus\partial)$ либо одновременно не определены, либо одновременно определены и равны.\\

\textbf{Свойство 1}: $\partial$ всегда определён и в худшем случае равен $\sigma$.\\

\textbf{Свойство 2}: $\partial\dfrac{\omega}{\omega}\sigma = \varnothing$.\\

\begin{comment}
	Вопрос: верно ли, что в условиях инкрементального случая (там, где $\Delta^\rho_\alpha\sigma$)

	$$\partial\dfrac{\omega_\rho}{\omega_\alpha}(\sigma\setminus\rho)\subseteq\xi$$ ?

	\textbf{Доказательство:}

	\newcommand{\mypart}{\partial\dfrac{\omega_\rho}{\omega_\alpha}(\sigma\setminus\rho)}

	Докажем, что если $s \in \sigma\setminus\rho$, $s \notin \xi$, то $s \notin \mypart$. Обозначим $\tau = (\sigma\setminus\rho)\setminus\mypart$, тогда $\tau$ --- наибольшее подмножество $\sigma\setminus\rho$, такое, что $gen(\omega_\rho, \tau)$ определено $\Leftrightarrow$ $gen(\omega_\alpha, \tau)$ определено. Предположим, $s \in \mypart$, тогда $s \notin \tau$. Рассмотрим $gen(\omega_\rho, \tau)$.
\end{comment}


\hrulefill

\textbf{Определение.} Пусть $\omega$, $\tilde{\omega}$ --- множества выходов, $\sigma$ --- множество входов. Известно, что определено $gen(\omega, \sigma) = \omega^\prime$, $gen(\tilde{\omega}, \sigma) = \tilde{\omega}^\prime$. Тогда $\partial\dfrac{\omega}{\tilde{\omega}}\sigma = \partial$ --- это наименьшее подмножество $\sigma$, удовлетворяющее свойству: 
$gen(\omega \cup \omega^\prime_{\partial}, \sigma\setminus\partial)$ и
$gen(\tilde{\omega} \cup \tilde{\omega}^\prime_{\partial}, \sigma\setminus\partial)$ либо одновременно не определены, либо одновременно определены и равны. 
\\

2. Свойство: $\partial$ всегда определен и (на худой конец) равен $\sigma$.
\\

3. Свойство: $\partial\dfrac{\omega}{\omega}\sigma = \varnothing$.
\\

Вопрос: верно ли, что в условиях инкрементального случая (там, где $\Delta^\rho_\alpha\sigma$)

$$\partial\dfrac{\omega_\rho}{\omega_\alpha}(\sigma\setminus\rho)\subseteq\xi$$ ?

\begin{comment}
	\textbf{Доказательство:}

	\newcommand{\mypart}{\partial\dfrac{\omega_\rho}{\omega_\alpha}(\sigma\setminus\rho)}

	Докажем, что если $s \in \sigma\setminus\rho$, $s \notin \xi$, то $s \notin \mypart$. Обозначим $\tau = (\sigma\setminus\rho)\setminus\mypart$, тогда $\tau$ --- наибольшее подмножество $\sigma\setminus\rho$, такое, что $gen(\omega_\rho, \tau)$ определено $\Leftrightarrow$ $gen(\omega_\alpha, \tau)$ определено. Предположим, $s \in \mypart$, тогда $s \notin \tau$. Рассмотрим $gen(\omega_\rho, \tau)$.
\end{comment}

\newpage

% ------------------------------------------------------
\newcommand{\butpartial}{\sigma\setminus\rho\setminus\partial}

\textbf{Теорема 1 (об инкрементальной компиляции).}

Пусть дано: $\sigma \subset \Sigma$, $gen(\varnothing, \sigma) = \omega^\sigma$. Пусть $\rho, \alpha \subset \Sigma$, при этом $\rho \subseteq \sigma$, $\sigma \cap \alpha = \varnothing$; $\Delta = \Delta^\rho_\alpha\sigma = \sigma\setminus\rho\cup\alpha$. Известно, что определено $gen(\omega^\sigma_{\sigma\setminus\rho}, \alpha) = \omega_\alpha$ и $gen(\varnothing, \Delta) = \omega^\Delta$. Обозначим $\partial = \partial\dfrac{\omega^\sigma_\rho}{\omega_\alpha}(\sigma\setminus\rho)$.
Тогда:

$$gen(\varnothing, \Delta) = \omega^\sigma_{\butpartial} \cup \omega_\alpha \cup gen(\omega^\sigma_{\butpartial} \cup \omega_\alpha, \partial)$$

\textbf{Доказательство:}
Рассмотрим $\omega^\Delta_\alpha = gen(\omega^\Delta_{\sigma\setminus\rho}, \alpha)$ и $\omega_\alpha = gen(\omega^\sigma_{\sigma\setminus\rho}, \alpha)$. Поскольку $B^0_{\omega^\Delta_{\sigma\setminus\rho}} = B^0_{\omega^\sigma_{\sigma\setminus\rho}}$, то по третьей аксиоме $\omega^\Delta_\alpha = \omega_\alpha$.

Докажем, что дифференциал $\partial = \partial\dfrac{\omega^\sigma_\rho}{\omega_\alpha}(\sigma\setminus\rho)$ имеет смысл. Для этого проверим, что определены $gen(\omega^\sigma_\rho, \sigma\setminus\rho)$ и $gen(\omega_\alpha, \sigma\setminus\rho)$. Первое выражение~--- это $\omega^\sigma_{\sigma\setminus\rho}$ и определено по аксиоме о дизъюнктном разбиении. Второе выражение равно $gen(\omega^\Delta_\alpha, \sigma\setminus\rho) = \omega^\Delta_{\sigma\setminus\rho}$ (как значения $gen$ для одного и того же входа с равными контекстами) и тоже определено по аксиоме о дизъюнктном разбиении. Тогда дифференциал в нашем случае, действительно, имеет смысл.

Рассмотрим равенство $gen(\omega_\alpha, \sigma\setminus\rho) = gen(\omega^\Delta_\alpha, \sigma\setminus\rho)$ и порождения, получающиеся в левой и в правой частях из $\partial$. В левой части обозначим их $\omega^\prime_\partial = gen(\omega_\alpha \cup \omega^{\sigma\setminus\rho}_{\butpartial}, \partial)$, а в правой части это $\omega^\Delta_\partial = gen(\omega^\Delta_\alpha \cup \omega^\Delta_{\butpartial}, \partial)$. Поскольку $B^0_{\omega_\alpha \cup \omega^{\sigma\setminus\rho}_{\butpartial}} = B^0_{\omega^\Delta_\alpha \cup \omega^\Delta_{\butpartial}}$, то по третьей аксиоме $\omega^\prime_\partial = \omega^\Delta_\partial$.

Рассмотрим $\butpartial$. По свойству дифференциала известно, что имеет место равенство $gen(\omega^\sigma_\rho \cup \omega^\sigma_\partial, \butpartial) = gen(\omega_\alpha \cup \omega^\prime_\partial, \butpartial)$ (в случае, если хотя бы одна из частей равенства определена). Левая часть этого равенства равна $\omega^\sigma_{\butpartial}$ и определена по аксиоме о дизъюнктном разбиении, а правая часть вследствие равенств $\omega_\alpha = \omega^\Delta_\alpha$ и $\omega^\prime_\partial = \omega^\Delta_\partial$ равна $gen(\omega^\Delta_\alpha \cup \omega^\Delta_\partial, \butpartial)$, то есть $\omega^\Delta_{\butpartial}$. Получаем, что $\omega^\sigma_{\butpartial} = \omega^\Delta_{\butpartial}$.

Рассмотрим $gen(\omega^\sigma_{\butpartial} \cup \omega_\alpha, \partial)$ из нашего утверждения. Вследствие равенств $\omega^\sigma_{\butpartial} = \omega^\Delta_{\butpartial}$ и $\omega_\alpha = \omega^\Delta_\alpha$ оно равно $gen(\omega^\Delta_{\butpartial} \cup \omega^\Delta_\alpha, \partial)$, то есть $\omega^\Delta_\partial$.

Рассмотрим элемент $x$ из левой части утверждения: $x \in \omega^\Delta$. Так как $\Delta = \sigma\setminus\rho\cup\alpha$, то по аксиоме о дизъюнктном разбиении верно, что $x$ принадлежит одному из трёх множеств: a) $\omega^\Delta_\alpha$; b) $\omega^\Delta_{\butpartial}$; c) $\omega^\Delta_\partial$. В случае a) из равенства $\omega^\Delta_\alpha = \omega_\alpha$ следует, что $x$ принадлежит и правой части. В случае b) из равенства $\omega^\Delta_{\butpartial} = \omega^\sigma_{\butpartial}$ следует, что $x$ принадлежит и правой части. В случае c) из равенства $\omega^\Delta_\partial = gen(\omega^\sigma_{\butpartial} \cup \omega_\alpha, \partial)$ следует, что $x$ принадлежит и правой части.

Рассмотрим элемент $x$ из правой части утверждения. Верно, что $x$ принадлежит хотя бы одному из трёх множеств (на самом деле только одному): a) $\omega_\alpha$; b) $\omega^\sigma_{\butpartial}$; c) $gen(\omega^\sigma_{\butpartial} \cup \omega_\alpha, \partial)$. В случае a) из равенства $\omega_\alpha = \omega^\Delta_\alpha$ следует, что $x$ принадлежит и левой части. В случае b) из равенства $\omega^\sigma_{\butpartial} = \omega^\Delta_{\butpartial}$ следует, что $x$ принадлежит и левой части. В случае c) из равенства $gen(\omega^\sigma_{\butpartial} \cup \omega_\alpha, \partial) = \omega^\Delta_\partial$ следует, что $x$ принадлежит и левой части.

Таким образом, правая и левая части утверждения совпадают и равенство действительно имеет место.

% ------------------------------------------------------
\newpage

\textbf{Теорема 2 (о переиспользовании порождений).}

Пусть $\forall i \in [1:n]$ дано: $\sigma_i$, $\omega_i = gen(\varnothing, \sigma_i)$, $\sigma_i^\prime \subseteq \sigma_i$ ($\sigma_i^\prime \cap \sigma_j^\prime = \varnothing$ при $i \neq j$). Обозначим $\omega_i^\prime = gen_i(\sigma_i^\prime)$. Обозначим 
$$\partial_i = \partial\dfrac{\omega_i \setminus \omega_i^\prime}{\bigcup\limits_{j \neq i} \omega_j^\prime} \sigma_i^\prime$$
Тогда:
$$gen(\varnothing, \bigcup\limits_i \sigma_i^\prime) = \left( \bigcup\limits_i \omega_i^\prime \setminus \omega_{\partial_i} \right) \cup gen(\bigcup\limits_i \omega_i^\prime \setminus \omega_{\partial_i}, \bigcup\limits_i \partial_i)$$

\newcommand{\sigi}{\sigma_i}
\newcommand{\sigpi}{\sigma^\prime_i}
\newcommand{\alloth}{\bigcup\limits_{j \neq i}\omega^\prime_j}

\textbf{Доказательство:}
Докажем, что дифференциал $\partial_i = \partial\dfrac{\omega_i \setminus \omega_i^\prime}{\bigcup\limits_{j \neq i} \omega_j^\prime} \sigma_i^\prime$ имеет смысл. Для этого проверим, что определены $gen(\omega_{\sigi} \setminus \omega_{\sigpi}, \sigpi)$ и $gen(\alloth, \sigpi)$. Поскольку определено $gen(\varnothing, \sigi)$, то по аксиоме о дизъюнктном разбиении определено и $gen(\omega_{\sigi} \setminus \omega_{\sigpi}, \sigpi)$. TODO: почему определено $gen(\alloth, \sigpi)$?



\end{document}