\documentclass[a4paper,12pt]{report} %%%{article}

\usepackage{cmap} % searchable PDFs
\usepackage[T2A]{fontenc} % scalable fonts
\usepackage[utf8]{inputenc} % input in UTF8
\usepackage[english,russian]{babel} % dashes on linebreaks
\usepackage{indentfirst} % indents in paragraphs
\usepackage{amstext,amssymb,amsfonts,amsmath,mathtext,enumerate,float}
\usepackage[left=25mm,right=2cm,top=2cm,bottom=2cm,bindingoffset=0cm]{geometry}
\usepackage[unicode]{hyperref}
\usepackage{graphicx}
\usepackage{ulem} % strikethrough
\usepackage{verbatim} % multiline comments
\usepackage{hhline}

\usepackage{listings}
\usepackage{color}
 
\lstset{extendedchars=\true}
  
\definecolor{dkgreen}{rgb}{0,0.6,0}
\definecolor{gray}{rgb}{0.5,0.5,0.5}
\definecolor{mauve}{rgb}{0.58,0,0.82}
 
\lstset{ %
  columns=flexible,
%  language=C,                     % the language of the code
  basicstyle=\footnotesize\ttfamily,       % the size of the fonts that are used for the code
  numbers=left,                   % where to put the line-numbers
  numberstyle=\tiny\color{gray},  % the style that is used for the line-numbers
  stepnumber=1,                   % the step between two line-numbers. If it's 1, each line will be numbered
  numbersep=5pt,                  % how far the line-numbers are from the code
  backgroundcolor=\color{white},  % choose the background color. You must add \usepackage{color}
  showspaces=false,               % show spaces adding particular underscores
  showstringspaces=false,         % underline spaces within strings
  showtabs=false,                 % show tabs within strings adding particular underscores
%  frame=single,                   % adds a frame around the code
  rulecolor=\color{black},        % if not set, the frame-color may be changed on line-breaks within not-black text (e.g. comments (green here))
  tabsize=4,                      % sets default tabsize
  captionpos=b,                   % sets the caption-position to bottom
  breaklines=true,                % sets automatic line breaking
  breakatwhitespace=false,        % sets if automatic breaks should only happen at whitespace
  title=\lstname,                 % show the filename of files included with \lstinputlisting; also try caption instead of title
  keywordstyle=\color{blue},      % keyword style
  commentstyle=\color{dkgreen},   % comment style
  stringstyle=\color{mauve},      % string literal style
%  escapeinside={\%*}{*)},         % if you want to add LaTeX within your code
  mathescape=true,
  morekeywords={*,...},           % if you want to add more keywords to the set
  deletekeywords={...},           % if you want to delete keywords from the given language
  aboveskip=1em,
  belowskip=-1em
}


\usepackage{mathtools}
\newcommand{\defeq}{\vcentcolon=}
\newcommand{\eqdef}{=\vcentcolon}

\renewcommand{\contentsname}{Содержание} 
\setcounter{secnumdepth}{0}
\setcounter{tocdepth}{3}

% \linespread{1.3}
\sloppy % align=justify


\begin{document}

\subsection{Задача переиспользования порождений других входов}

Пусть есть $\sigma_1, \dots, \sigma_n$ ($\sigma_i \subset \Sigma$)~--- наборы входов и известны $\omega_1, \dots, \omega_n$, где $\omega_i = gen^d(\sigma_i)$.\\
Также есть набор $\sigma^*$, при этом $\sigma^* \subset \bigcup\limits_{i \in [1:n]}\sigma_i$, т.е. $\forall s \in \sigma^*$ $\exists i_s \in [1:n]: s \in \sigma_{i_s}$.\\
Требуется получить $gen^d(\sigma^*)$, используя известные порождения $\omega_1, \dots, \omega_n$.\\

Пусть

$$\sigma_i' = \sigma_i \cap \sigma^*,$$
$$\rho_i = \sigma_i \setminus \sigma^* = \sigma_i \setminus \sigma_i',$$
$$\alpha_i = \sigma^* \setminus \sigma_i = \sigma^* \setminus \sigma_i',$$
$$P_i = \{s \in \sigma_i \mid s \in \rho_i \vee d(\sigma_i, s) \cap \rho_i \neq \varnothing \},$$

\begin{comment}
$$A_i = \{s \mid s \in \alpha_i \vee d(\sigma^*, s) \cap \alpha_i \neq \varnothing \} \cup (P_i \setminus \rho_i),$$

	тогда
	$$gen^d(\sigma^*) = \left( \bigcup\limits_{i = 1}^n \left( gen^d(\sigma_i) \setminus \bigcup\limits_{s \in P_i} gen_{d(\sigma_i, s)}(s)\right) \right) \cup \bigcup\limits_{s \in \bigcap\limits_{i = 1}^n A_i} gen_{d(\sigma^*, s)}(s)$$
\end{comment}

\hrulefill

\begin{comment}
	\textbf{Доказательство:}\\

	Пусть некоторый $x \in gen^d(\sigma^*)$. Это означает, что $\exists s_x \in \sigma^*$: $x \in gen_{d(\sigma^*, s_x)} (s_x)$.\\
	Нужно доказать, что $x$ принадлежит правой части, то есть:
	$$
	\left[
	\begin{aligned}
		&\exists a \in \sigma_i \setminus P_i: &x \in gen_{d(\sigma_i, a)} (a)\\
		&\exists b: \forall i \; b \in A_i, &x \in gen_{d(\sigma^*, b)} (b)\\
	\end{aligned}
	\right.
	$$

	Рассмотрим два случая:
	a) $\exists i : s_x \in \sigma_i \setminus P_i, $
\end{comment}

Введём обозначения: $gen_i(s) = gen_{d(\sigma_i, s)}(s)$, $gen^*(s) = gen_{d(\sigma^*, s)}(s)$.\\
Рассмотрим различные случаи для $s \in \sigma_i'$ и действия, которые необходимо произвести с $gen^d(\sigma_i)$, чтобы получить $gen^d(\sigma^*)$: \\

\begin{tabular}{ | p{3.5cm} || p{3.5cm} | p{3.5cm} | p{3.5cm} |}
	\hline
	Действие & $d(\sigma_i, s) \cap \rho_i \neq \varnothing$ & $d(\sigma_i, s) \cap \rho_i = \varnothing$ & $d(\sigma_i, s)$ не опр. \\ \hline
	& & & \\[-1.1em] \hline
	$d(\sigma^*, s) \cap \alpha_i \neq \varnothing$ & вычесть $gen_i(s)$, добавить $gen^*(s)$ & не бывает по предположению & добавить $gen^*(s)$ \\ \hline
	$d(\sigma^*, s) \cap \alpha_i = \varnothing$ & не бывает по предположению & взять $gen_i(s)$ & добавить $gen^*(s)$ \\ \hline
	$d(\sigma^*, s)$ не опр. & вычесть $gen_i(s)$ & вычесть $gen_i(s)$ & ничего не делать \\
	\hline
\end{tabular} \\\\

Введём ещё $Q_i = \{s \mid d(\sigma_i, s) \cap \rho_i = \varnothing$ \& $d(\sigma^*, s) \cap \alpha_i = \varnothing\}$ (центральная ячейка). На самом деле $Q_i = \sigma_i \setminus P_i \setminus \{s \mid d(\sigma^*, s) \text{ не опр.}\} \setminus \{s \mid d(\sigma_i, s) \text{ не опр.}\}$.

Тогда рассмотрим произвольное $i$ и произвольное $s \in \sigma^*$. Либо $s \in \alpha_i$, либо $s \in \sigma_i'$. Во втором случае либо $s \in Q_i$, и тогда по допущению b) $gen^*(s) =  gen_i(s)$; либо нет, и тогда мы не можем использовать $gen_i(s)$.

Теперь рассмотрим произвольное $s \in \sigma^*$. Либо $\exists i_s$ : $s \in Q_i$, либо $\forall i \in [1:n]$ : $s \in \alpha_i \cup (\sigma_i' \setminus Q_i) = \alpha_i \cup \{s | d(\sigma_i, s) \cap \rho_i \neq \varnothing \vee d(\sigma_i, s) \text{ не опр.} \vee d(\sigma^*, s) \cap \alpha_i \neq \varnothing \vee d(\sigma^*, s) \text{ не опр.} \}$.

Во втором случае~--- если принадлежит всем, то принадлежит и пересечению:
$$s \in \bigcap\limits_{i = 1}^n \alpha_i \cup (\sigma_i' \setminus Q_i),$$
и тогда 
$$gen^*(s) \in \bigcup\limits_{t \in \bigcap\limits_{i = 1}^n \alpha_i \cup (\sigma_i' \setminus Q_i)} gen^*(t).$$
Для $s$ таких, что $d(\sigma^*, s)$ не определено, $gen^*(s) = \varnothing$ независимо от $i$, поэтому можно исключить их из рассмотрения. Тогда получается, что пересекать достаточно только множества $A_i$, где
$$A_i = \alpha_i \cup \{s \in \sigma_i' \mid (d(\sigma_i, s) \cap \rho_i \neq \varnothing \;\&\; d(\sigma^*, s) \cap \alpha_i \neq \varnothing) \vee (d(\sigma_i, s) \text{ не опр.} \;\&\; d(\sigma^*, s) \text{ опред.})\}.$$

Но ничего не изменится (см. таблицу), если в качестве $A_i$ рассматривать
$$A_i = \alpha_i \cup \{s \in \sigma_i\setminus\rho_i \mid d(\sigma_i, s) \cap \rho_i \neq \varnothing \vee d(\sigma_i, s) \text{ не опр.}\}$$
или
$$A_i = \alpha_i \cup \{s \in \sigma_i\setminus\rho_i \mid d(\sigma^*, s) \cap \alpha_i \neq \varnothing \vee d(\sigma_i, s) \text{ не опр.}\}$$

Отсюда гипотеза:
$$gen^d(\sigma^*) = \left( \bigcup\limits_{i = 1}^n \bigcup\limits_{s \in Q_i} gen_{d(\sigma_i, s)}(s) \right) \cup \bigcup\limits_{s \in \bigcap\limits_{i = 1}^n A_i} gen_{d(\sigma^*, s)}(s)$$
\end{document}