\documentclass[a4paper,12pt]{report} %%%{article}

\usepackage{cmap} % searchable PDFs
\usepackage[T2A]{fontenc} % scalable fonts
\usepackage[utf8]{inputenc} % input in UTF8
\usepackage[english,russian]{babel} % dashes on linebreaks
\usepackage{indentfirst} % indents in paragraphs
\usepackage{amstext,amssymb,amsfonts,amsmath,mathtext,enumerate,float}
\usepackage[left=25mm,right=2cm,top=2cm,bottom=2cm,bindingoffset=0cm]{geometry}
\usepackage[unicode]{hyperref}
\usepackage{graphicx}
\usepackage{ulem} % strikethrough
\usepackage{verbatim} % multiline comments
\usepackage{hhline}

\usepackage{listings}
\usepackage{color}
 
\lstset{extendedchars=\true}
  
\definecolor{dkgreen}{rgb}{0,0.6,0}
\definecolor{gray}{rgb}{0.5,0.5,0.5}
\definecolor{mauve}{rgb}{0.58,0,0.82}
 
\lstset{ %
  columns=flexible,
%  language=C,                     % the language of the code
  basicstyle=\footnotesize\ttfamily,       % the size of the fonts that are used for the code
  numbers=left,                   % where to put the line-numbers
  numberstyle=\tiny\color{gray},  % the style that is used for the line-numbers
  stepnumber=1,                   % the step between two line-numbers. If it's 1, each line will be numbered
  numbersep=5pt,                  % how far the line-numbers are from the code
  backgroundcolor=\color{white},  % choose the background color. You must add \usepackage{color}
  showspaces=false,               % show spaces adding particular underscores
  showstringspaces=false,         % underline spaces within strings
  showtabs=false,                 % show tabs within strings adding particular underscores
%  frame=single,                   % adds a frame around the code
  rulecolor=\color{black},        % if not set, the frame-color may be changed on line-breaks within not-black text (e.g. comments (green here))
  tabsize=4,                      % sets default tabsize
  captionpos=b,                   % sets the caption-position to bottom
  breaklines=true,                % sets automatic line breaking
  breakatwhitespace=false,        % sets if automatic breaks should only happen at whitespace
  title=\lstname,                 % show the filename of files included with \lstinputlisting; also try caption instead of title
  keywordstyle=\color{blue},      % keyword style
  commentstyle=\color{dkgreen},   % comment style
  stringstyle=\color{mauve},      % string literal style
%  escapeinside={\%*}{*)},         % if you want to add LaTeX within your code
  mathescape=true,
  morekeywords={*,...},           % if you want to add more keywords to the set
  deletekeywords={...},           % if you want to delete keywords from the given language
  aboveskip=1em,
  belowskip=-1em
}


\usepackage{mathtools}
\newcommand{\defeq}{\vcentcolon=}
\newcommand{\eqdef}{=\vcentcolon}

\renewcommand{\contentsname}{Содержание} 
\setcounter{secnumdepth}{0}
\setcounter{tocdepth}{3}

% \linespread{1.3}
\sloppy % align=justify


\begin{document}

\subsection{Задача инкрементальной компиляции}

\newcommand{\Deltasigma}{\Delta_\alpha^\rho\sigma}
Рассмотрим $\Deltasigma = (\sigma \setminus \rho) \cup \alpha$, где $\rho, \alpha \subset \Sigma$, при этом $\rho \subseteq \sigma$, $\sigma \cap \alpha = \varnothing$. Тогда
$$gen^d(\Deltasigma) \overset{def}{=} \bigcup\limits_{s \in \Deltasigma} gen_{d(\Deltasigma, s)}(s) = \bigcup\limits_{s \in (\sigma \setminus \rho) \cup \alpha} gen_{d((\sigma \setminus \rho) \cup \alpha, s)}(s).$$
Требуется выразить эту функцию через $gen^d(\sigma)$.

Гипотеза: пусть
$$P = \{s \in \sigma \mid s \in \rho \vee d(\sigma, s) \cap \rho \neq \varnothing \},$$
$$A = \{s \mid s \in \alpha \vee d(\Deltasigma, s) \cap \alpha \neq \varnothing \} \cup (P \setminus \rho),$$
тогда
$$gen^d(\Deltasigma) = gen^d(\sigma) \setminus \bigcup\limits_{s \in P} gen_{d(\sigma, s)}(s) \cup \bigcup\limits_{s \in A} gen_{d(\Deltasigma, s)}(s)$$

\hrulefill

Я пытался проверить гипотезу, доказывая, что левая часть последнего утверждения входит в правую и наоборот. В одну сторону всё хорошо доказывается, а в обратную есть следующая проблема.

Рассмотрим элементы $s \in \sigma \setminus P$, у которых $d(\Deltasigma, s) \cap \alpha \neq \varnothing$ (то есть эти элементы не менялись, у них нет зависимостей в $\rho$, но есть зависимости в $\alpha$). В контексте $\sigma$ они не должны были компилироваться из-за отсутствующих зависимостей (то есть вычисление $d(\sigma, s)$ для них некорректно~--- нет такого подмножества $\sigma$, которое являлось бы необходимым контекстом), а в новом контексте $\Deltasigma$ их ``неудовлетворённые'' зависимости, лежащие в $\alpha$, удовлетворяются.

В текущей формулировке гипотезы те порождения, которые получились после неудачной компиляции таких элементов в контексте $\sigma$, никак не исключаются из множества $gen^d(\sigma)$, как это делается с $s \in P$. Поэтому в правой части эти порождения присутствуют, а в левой, понятно,~--- нет.

Это, видимо, решается пополнением $P$:
$$P = \rho \cup \{ s \in \sigma \mid d(\sigma, s) \cap \rho \neq \varnothing \vee d(\Deltasigma, s) \cap \alpha \neq \varnothing \}$$
Заодно можно в похожей форме переписать $A$:
$$A = \alpha \cup \{ s \in \Deltasigma \mid d(\sigma, s) \cap \rho \neq \varnothing \vee d(\Deltasigma, s) \cap \alpha \neq \varnothing \} \setminus \rho$$
Тогда, в самом деле, гипотеза верна с некоторыми допущениями:\\
\indent a) для произвольных $\sigma$, $x$ верно, что $(\exists s: x \in gen_\sigma(s))$ $\Rightarrow$ $\forall t \neq s$ $x \notin gen_\sigma(t)$ (в одном и том же контексте порождение может принадлежать результату генерации для не более чем одного входа)\\
\indent b) $\forall s \in \{s \in \sigma \setminus \rho \mid d(\sigma, s) \cap \rho = \varnothing \wedge d(\Deltasigma, s) \cap \alpha = \varnothing \}$ $gen_{d(\sigma, s)} (s) = gen_{d(\Deltasigma, s)} (s)$ (если ни сам вход, ни его зависимости не лежали в изменяющейся части, результат генерации не меняется).\\

\textbf{Доказательство:}\\
Пусть некоторый $x \in gen^d(\Deltasigma)$. Это означает, что $\exists s_x \in (\sigma \setminus \rho) \cup \alpha$: $x \in gen_{d(\Deltasigma, s_x)} (s_x)$.\\
Нужно доказать, что $x$ принадлежит правой части, то есть:
$$
\left[
\begin{aligned}
	&\left\{
	\begin{aligned}
		\exists a \in \sigma&: x \in gen_{d(\sigma, a)} (a) &(1)\\
		\forall b \in P&: x \notin gen_{d(\sigma, b)} (b) &(2)\\
		% TODO: make normal enumeration here
	\end{aligned}
	\right.\\
	&\exists c \in A: x \in gen_{d(\Deltasigma, c)} (c)\\
\end{aligned}
\right.
$$

На самом деле систему можно переписать:

$$
\left[
\begin{aligned}
	&\exists a \in \sigma \setminus P: &x \in gen_{d(\sigma, a)} (a)\\
	&\exists c \in A: &x \in gen_{d(\Deltasigma, c)} (c)\\
\end{aligned}
\right.
$$

В силу первого допущения верность первого утверждения обеспечит верность для (1) и (2).\\
Рассмотрим два случая:\\
\indent a) $s_x \in \alpha$. Тогда предъявим $c = s_x$; $c \in \alpha \subseteq A$ и, действительно, $x \in gen_{d(\Deltasigma, c)} (c)$.\\
\indent b) $s_x \in \sigma \setminus \rho$. Тогда если $d(\sigma, s_x) \cap \rho \neq \varnothing$ или $d(\Deltasigma, s_x) \cap \alpha \neq \varnothing$, то опять предъявим $c = s_x \in A$. В противном случае воспользуемся вторым допущением и предъявим $a = s_x$.\\

Обратно: пусть некоторый $x$ принадлежит правой части. Нужно доказать, что $x$ принадлежит левой части, то есть $\exists s_x \in \Deltasigma$: $x \in gen_{d(\Deltasigma, s_x)} (s_x)$.\\
Рассмотрим два случая:\\
\indent a) $\exists c \in A: x \in gen_{d(\Deltasigma, c)} (c)$. Ясно, что $A \subseteq \Deltasigma$. Тогда предъявим $s_x = c \in \Deltasigma$. Действительно, $x \in gen_{d(\Deltasigma, s_x)} (s_x)$.\\
\indent b) $\forall c \in A: x \notin gen_{d(\Deltasigma, c)} (c)$, но $\exists a \in \sigma \setminus P: x \in gen_{d(\sigma, a)} (a)$. Ясно, что и $\sigma \setminus P \subseteq \Deltasigma$. Тогда предъявим $s_x = a \in \Deltasigma$. Действительно, по второму допущению $gen_{d(\sigma, a)} (a) = gen_{d(\Deltasigma, s_x)} (s_x)$ и $x \in gen_{d(\Deltasigma, s_x)} (s_x)$.\\

Конец.

\end{document}