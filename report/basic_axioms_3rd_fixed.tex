\section{Основные определения и аксиоматика}

$\Sigma$ --- множество входов, $\Omega$ --- множество выходов (н.б.ч.с.). Функция порождения
выходов по входам (частичная):

$$
gen : 2^\Omega\to 2^\Sigma\to 2^\Omega
$$

Введём разбиение результата $gen(\omega, \sigma) = \omega^\prime$ на классы. Класс $B^0_{\omega^\prime}$ - это те порождения или части порождений, которые зависят только от входа $\sigma$. Класс $B^1_{\omega^\prime}$ - те порождения или части порождений, которые зависят только от входа и порождений класса $B^0_{\omega}$. По индукции, класс $B^i_{\omega^\prime}$ - те порождения или части порождений, которые зависят только от входа и порождений классов $B^j_{\omega}$, где $j < i$.\\

Аксиомы функции порождения:

\begin{enumerate}
	\item если $gen(\omega,\sigma)$ --- определена, то $\omega\cap gen(\omega,\sigma)=\varnothing$;
	
	\item если $gen(\omega,\sigma)=\omega^\prime$, то существует единственное дизъюнктное разбиение $\omega^\prime=\bigcup^\varnothing_{s\in\sigma}\omega^\prime_s$, 
	удовлетворяющее свойству 

	$$\forall s\in\sigma : gen(\omega\cup\omega^\prime\setminus\omega^\prime_s,\{s\})=\omega^\prime_s$$

	\item $\forall s\in\Sigma,\; \forall\omega,\:\omega^\prime\subseteq\Omega:$ если $s$ невырожденное, $gen(\omega,\{s\})$ определено и $B^0_{\omega} = B^0_{\omega^\prime}$, то и $gen(\omega^\prime,\{s\})$ определено, более того, $gen(\omega,\{s\}) = gen(\omega^\prime,\{s\})$;

	\item если $gen(\omega,\{s\})$ --- определена, то в $\omega$ существует наименьшее по включению подмножество $d_\omega(s)$, такое, что
	$gen(d_\omega(s), \{s\})$ --- определена;

	\item если $gen(\omega,\sigma)=\omega^\prime$ и для какого-то $s\in\sigma$ существует $s_1\in\sigma$, такой, что
	% $d_\omega(s)\cap \omega^\prime_{s_1}\ne\varnothing$, 
	$d_{\omega\cup\omega^\prime\setminus\omega^\prime_s}(s) \cap \omega^\prime_{s_1}\ne\varnothing$, 
	то $gen(\omega,\sigma\setminus s_1)$ --- не определена;
	
	\item если $gen(\omega,\{s\})$ --- определено, то для произвольного $\omega^\prime\subseteq\omega$, такого, что $d_\omega(s)\subseteq\omega^\prime$, $gen(\omega^\prime, \{s\})$ тоже определено и равно $gen(\omega,\{s\})$.
\end{enumerate}