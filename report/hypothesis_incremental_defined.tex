\documentclass[a4paper,12pt]{report} %%%{article}

\usepackage{cmap} % searchable PDFs
\usepackage[T2A]{fontenc} % scalable fonts
\usepackage[utf8]{inputenc} % input in UTF8
%\usepackage[english,russian]{babel} % dashes on linebreaks
\usepackage{indentfirst} % indents in paragraphs
\usepackage{amstext,amssymb,amsfonts,amsmath,mathtext,enumerate,float}
\usepackage[left=25mm,right=2cm,top=2cm,bottom=2cm,bindingoffset=0cm]{geometry}
\usepackage[unicode]{hyperref}
\usepackage{graphicx}
\usepackage{ulem} % strikethrough
\usepackage{verbatim} % multiline comments
\usepackage{hhline}

\usepackage{listings}
\usepackage{color}
 
\definecolor{dkgreen}{rgb}{0,0.6,0}
\definecolor{gray}{rgb}{0.5,0.5,0.5}
\definecolor{mauve}{rgb}{0.58,0,0.82}
 
\lstset{ %
  columns=flexible,
%  language=C,                     % the language of the code
  basicstyle=\footnotesize\ttfamily,       % the size of the fonts that are used for the code
  numbers=left,                   % where to put the line-numbers
  numberstyle=\tiny\color{gray},  % the style that is used for the line-numbers
  stepnumber=1,                   % the step between two line-numbers. If it's 1, each line 
                                  % will be numbered
  numbersep=5pt,                  % how far the line-numbers are from the code
  backgroundcolor=\color{white},  % choose the background color. You must add \usepackage{color}
  showspaces=false,               % show spaces adding particular underscores
  showstringspaces=false,         % underline spaces within strings
  showtabs=false,                 % show tabs within strings adding particular underscores
%  frame=single,                   % adds a frame around the code
  rulecolor=\color{black},        % if not set, the frame-color may be changed on line-breaks within not-black text (e.g. comments (green here))
  tabsize=4,                      % sets default tabsize
  captionpos=b,                   % sets the caption-position to bottom
  breaklines=true,                % sets automatic line breaking
  breakatwhitespace=false,        % sets if automatic breaks should only happen at whitespace
  title=\lstname,                 % show the filename of files included with \lstinputlisting;
                                  % also try caption instead of title
  keywordstyle=\color{blue},      % keyword style
  commentstyle=\color{dkgreen},   % comment style
  stringstyle=\color{mauve},      % string literal style
%  escapeinside={\%*}{*)},         % if you want to add LaTeX within your code
  mathescape=true,
  morekeywords={*,...},           % if you want to add more keywords to the set
  deletekeywords={...}            % if you want to delete keywords from the given language
}

\lstset{
literate={а}{{\selectfont\char224}}1
{б}{{\selectfont\char225}}1
{в}{{\selectfont\char226}}1
{г}{{\selectfont\char227}}1
{д}{{\selectfont\char228}}1
{е}{{\selectfont\char229}}1
{ё}{{\"e}}1
{ж}{{\selectfont\char230}}1
{з}{{\selectfont\char231}}1
{и}{{\selectfont\char232}}1
{й}{{\selectfont\char233}}1
{к}{{\selectfont\char234}}1
{л}{{\selectfont\char235}}1
{м}{{\selectfont\char236}}1
{н}{{\selectfont\char237}}1
{о}{{\selectfont\char238}}1
{п}{{\selectfont\char239}}1
{р}{{\selectfont\char240}}1
{с}{{\selectfont\char241}}1
{т}{{\selectfont\char242}}1
{у}{{\selectfont\char243}}1
{ф}{{\selectfont\char244}}1
{х}{{\selectfont\char245}}1
{ц}{{\selectfont\char246}}1
{ч}{{\selectfont\char247}}1
{ш}{{\selectfont\char248}}1
{щ}{{\selectfont\char249}}1
{ъ}{{\selectfont\char250}}1
{ы}{{\selectfont\char251}}1
{ь}{{\selectfont\char252}}1
{э}{{\selectfont\char253}}1
{ю}{{\selectfont\char254}}1
{я}{{\selectfont\char255}}1
{А}{{\selectfont\char192}}1
{Б}{{\selectfont\char193}}1
{В}{{\selectfont\char194}}1
{Г}{{\selectfont\char195}}1
{Д}{{\selectfont\char196}}1
{Е}{{\selectfont\char197}}1
{Ё}{{\"E}}1
{Ж}{{\selectfont\char198}}1
{З}{{\selectfont\char199}}1
{И}{{\selectfont\char200}}1
{Й}{{\selectfont\char201}}1
{К}{{\selectfont\char202}}1
{Л}{{\selectfont\char203}}1
{М}{{\selectfont\char204}}1
{Н}{{\selectfont\char205}}1
{О}{{\selectfont\char206}}1
{П}{{\selectfont\char207}}1
{Р}{{\selectfont\char208}}1
{С}{{\selectfont\char209}}1
{Т}{{\selectfont\char210}}1
{У}{{\selectfont\char211}}1
{Ф}{{\selectfont\char212}}1
{Х}{{\selectfont\char213}}1
{Ц}{{\selectfont\char214}}1
{Ч}{{\selectfont\char215}}1
{Ш}{{\selectfont\char216}}1
{Щ}{{\selectfont\char217}}1
{Ъ}{{\selectfont\char218}}1
{Ы}{{\selectfont\char219}}1
{Ь}{{\selectfont\char220}}1
{Э}{{\selectfont\char221}}1
{Ю}{{\selectfont\char222}}1
{Я}{{\selectfont\char223}}1
}
\usepackage{mathtools}
\newcommand{\defeq}{\vcentcolon=}
\newcommand{\eqdef}{=\vcentcolon}

\renewcommand{\contentsname}{Содержание} 
\setcounter{secnumdepth}{0}
\setcounter{tocdepth}{3}

%\sloppy % align=justify

\begin{document}

\subsection{Задача инкрементальной компиляции}

\newcommand{\Deltasigma}{\Delta_\alpha^\rho\sigma}
Рассмотрим $\Deltasigma = (\sigma \setminus \rho) \cup \alpha$, где $\rho, \alpha \subset \Sigma$, при этом $\rho \subseteq \sigma$, $\sigma \cap \alpha = \varnothing$. Тогда
$$gen^d(\Deltasigma) \overset{def}{=} \bigcup\limits_{s \in \Deltasigma} gen_{d(\Deltasigma, s)}(s) = \bigcup\limits_{s \in (\sigma \setminus \rho) \cup \alpha} gen_{d((\sigma \setminus \rho) \cup \alpha, s)}(s).$$
Требуется выразить эту функцию через $gen^d(\sigma)$.\\
Гипотеза: пусть
$$P = \{s \in \sigma \mid s \in \rho \vee d(\sigma, s) \cap \rho \neq \varnothing \},$$
$$A = \{s \mid s \in \alpha \vee d(\Deltasigma, s) \cap \alpha \neq \varnothing \} \cup (P \setminus \rho),$$
тогда
$$gen^d(\Deltasigma) = gen^d(\sigma) \setminus \bigcup\limits_{s \in P} gen_{d(\sigma, s)}(s) \cup \bigcup\limits_{s \in A} gen_{d(\Deltasigma, s)}(s)$$

Сделаем несколько допущений-аксиом:\\
\indent a) для произвольных $\sigma$, $x$ верно, что $(\exists s: x \in gen_\sigma(s))$ $\Rightarrow$ $\forall t \neq s$ $x \notin gen_\sigma(t)$ (в одном и том же контексте порождение может принадлежать результату генерации для не более чем одного входа)\\
\indent b) $\forall s \in \{s \in \sigma \setminus \rho \mid d(\sigma, s) \cap \rho = \varnothing \wedge d(\Deltasigma, s) \cap \alpha = \varnothing \}$ $gen_{d(\sigma, s)} (s) = gen_{d(\Deltasigma, s)} (s)$ (если ни сам вход, ни его зависимости не лежали в изменяющейся части, результат генерации не меняется)\\
\indent c) $d(\sigma, s)$ не определено, если для $s$ невозможно найти подмножество $\sigma$, являющееся необходимым и достаточным контекстом (из-за отсутствующих в контексте или дублирующих друг друга зависимостей); в таком случае $gen_{d(\sigma, s)}(s) = \varnothing$\\
\indent d) если $d(\sigma, s)$ определено, то $\forall \tau$ таких, что $\sigma \cap \tau = \varnothing$, $d(\sigma \cup \tau, s)$ либо равно $d(\sigma, s)$, либо не определено.

Гипотеза 2: пусть
$$\forall s \in \sigma \; d(\sigma, s) \; \text{определено},$$ 
тогда
$$\forall s \in \sigma \! \setminus \! P \; d(\Deltasigma, s) \cap \alpha = \varnothing$$

\hrulefill

Конец.

\end{document}