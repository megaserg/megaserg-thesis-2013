\section{Обзор}

\subsection{Инкрементальная компиляция в IntelliJ IDEA}
В текущей реализации инкрементальной компиляции в IntelliJ IDEA используется локальное кэширование, изменение файлов отслеживается по timestamp~--- для каждого файла запоминается время его самого позднего изменения на момент последней компиляции, и файл считается изменившимся, если фактическое время его самого позднего изменения не совпадает с запомненным. В кэше хранятся абсолютные пути к файлам, таким образом, при перемещении проекта или его части в другую директорию кэш теряется. В ситуации, когда разработчиков несколько, временные метки в силу их относительности использовать уже нельзя. Предлагается вместо них применять хэширование исходных файлов, а именно хэширующую структуру, способную считать контрольные суммы не только на уровне отдельных файлов, но и на уровне директорий и модулей проекта.

\subsection{Инкрементальная компиляция в Eclipse}

\subsection{Инструмент ccache}