\documentclass[a4paper,12pt]{report} %%%{article}

\usepackage{cmap} % searchable PDFs
\usepackage[T2A]{fontenc} % scalable fonts
\usepackage[utf8]{inputenc} % input in UTF8
%\usepackage[english,russian]{babel} % dashes on linebreaks
\usepackage{indentfirst} % indents in paragraphs
\usepackage{amstext,amssymb,amsfonts,amsmath,mathtext,enumerate,float}
\usepackage[left=25mm,right=2cm,top=2cm,bottom=2cm,bindingoffset=0cm]{geometry}
\usepackage[unicode]{hyperref}
\usepackage{graphicx}
\usepackage{ulem} % strikethrough
\usepackage{verbatim} % multiline comments
\usepackage{hhline}

\usepackage{listings}
\usepackage{color}
 
\definecolor{dkgreen}{rgb}{0,0.6,0}
\definecolor{gray}{rgb}{0.5,0.5,0.5}
\definecolor{mauve}{rgb}{0.58,0,0.82}
 
\lstset{ %
  columns=flexible,
%  language=C,                     % the language of the code
  basicstyle=\footnotesize\ttfamily,       % the size of the fonts that are used for the code
  numbers=left,                   % where to put the line-numbers
  numberstyle=\tiny\color{gray},  % the style that is used for the line-numbers
  stepnumber=1,                   % the step between two line-numbers. If it's 1, each line 
                                  % will be numbered
  numbersep=5pt,                  % how far the line-numbers are from the code
  backgroundcolor=\color{white},  % choose the background color. You must add \usepackage{color}
  showspaces=false,               % show spaces adding particular underscores
  showstringspaces=false,         % underline spaces within strings
  showtabs=false,                 % show tabs within strings adding particular underscores
%  frame=single,                   % adds a frame around the code
  rulecolor=\color{black},        % if not set, the frame-color may be changed on line-breaks within not-black text (e.g. comments (green here))
  tabsize=4,                      % sets default tabsize
  captionpos=b,                   % sets the caption-position to bottom
  breaklines=true,                % sets automatic line breaking
  breakatwhitespace=false,        % sets if automatic breaks should only happen at whitespace
  title=\lstname,                 % show the filename of files included with \lstinputlisting;
                                  % also try caption instead of title
  keywordstyle=\color{blue},      % keyword style
  commentstyle=\color{dkgreen},   % comment style
  stringstyle=\color{mauve},      % string literal style
%  escapeinside={\%*}{*)},         % if you want to add LaTeX within your code
  mathescape=true,
  morekeywords={*,...},           % if you want to add more keywords to the set
  deletekeywords={...}            % if you want to delete keywords from the given language
}

\lstset{
literate={а}{{\selectfont\char224}}1
{б}{{\selectfont\char225}}1
{в}{{\selectfont\char226}}1
{г}{{\selectfont\char227}}1
{д}{{\selectfont\char228}}1
{е}{{\selectfont\char229}}1
{ё}{{\"e}}1
{ж}{{\selectfont\char230}}1
{з}{{\selectfont\char231}}1
{и}{{\selectfont\char232}}1
{й}{{\selectfont\char233}}1
{к}{{\selectfont\char234}}1
{л}{{\selectfont\char235}}1
{м}{{\selectfont\char236}}1
{н}{{\selectfont\char237}}1
{о}{{\selectfont\char238}}1
{п}{{\selectfont\char239}}1
{р}{{\selectfont\char240}}1
{с}{{\selectfont\char241}}1
{т}{{\selectfont\char242}}1
{у}{{\selectfont\char243}}1
{ф}{{\selectfont\char244}}1
{х}{{\selectfont\char245}}1
{ц}{{\selectfont\char246}}1
{ч}{{\selectfont\char247}}1
{ш}{{\selectfont\char248}}1
{щ}{{\selectfont\char249}}1
{ъ}{{\selectfont\char250}}1
{ы}{{\selectfont\char251}}1
{ь}{{\selectfont\char252}}1
{э}{{\selectfont\char253}}1
{ю}{{\selectfont\char254}}1
{я}{{\selectfont\char255}}1
{А}{{\selectfont\char192}}1
{Б}{{\selectfont\char193}}1
{В}{{\selectfont\char194}}1
{Г}{{\selectfont\char195}}1
{Д}{{\selectfont\char196}}1
{Е}{{\selectfont\char197}}1
{Ё}{{\"E}}1
{Ж}{{\selectfont\char198}}1
{З}{{\selectfont\char199}}1
{И}{{\selectfont\char200}}1
{Й}{{\selectfont\char201}}1
{К}{{\selectfont\char202}}1
{Л}{{\selectfont\char203}}1
{М}{{\selectfont\char204}}1
{Н}{{\selectfont\char205}}1
{О}{{\selectfont\char206}}1
{П}{{\selectfont\char207}}1
{Р}{{\selectfont\char208}}1
{С}{{\selectfont\char209}}1
{Т}{{\selectfont\char210}}1
{У}{{\selectfont\char211}}1
{Ф}{{\selectfont\char212}}1
{Х}{{\selectfont\char213}}1
{Ц}{{\selectfont\char214}}1
{Ч}{{\selectfont\char215}}1
{Ш}{{\selectfont\char216}}1
{Щ}{{\selectfont\char217}}1
{Ъ}{{\selectfont\char218}}1
{Ы}{{\selectfont\char219}}1
{Ь}{{\selectfont\char220}}1
{Э}{{\selectfont\char221}}1
{Ю}{{\selectfont\char222}}1
{Я}{{\selectfont\char223}}1
}
\usepackage{mathtools}
\newcommand{\defeq}{\vcentcolon=}
\newcommand{\eqdef}{=\vcentcolon}

\renewcommand{\contentsname}{Содержание} 
\setcounter{secnumdepth}{0}
\setcounter{tocdepth}{3}

%\sloppy % align=justify

\begin{document}

Значок $=^k$ означает ``равенство с точностью до $k$-го класса'', т.е. $a =^k b$ означает, что для всех $i$ от $0$ до $k$ верно $B^i_a = B^i_b$.

\textbf{Равенство для нулевого класса:}

$$gen(\varnothing, \Delta) =^0 \omega^\sigma_{\sigma\setminus\rho} \cup gen(\omega^\sigma_{\sigma\setminus\rho}, \alpha) \eqdef \omega^\sigma_{\sigma\setminus\rho} \cup \omega_\alpha$$

\textbf{Равенство для первого класса:}

$$\partial_1 \defeq \partial\dfrac{\omega^\sigma_\rho}{\omega^\Delta_{\Delta\setminus(\sigma\setminus\rho)}}(\sigma\setminus\rho) = \partial\dfrac{\omega^\sigma_\rho}{\omega^\Delta_\alpha}(\sigma\setminus\rho)$$
$$gen(\varnothing, \Delta) =^1 \omega^\sigma_{\sigma\setminus\rho\setminus{\partial_1}} \cup \omega_\alpha \cup gen(\omega^\sigma_{\sigma\setminus\rho\setminus{\partial_1}} \cup \omega_\alpha, \partial_1) \eqdef \omega^\sigma_{\sigma\setminus\rho\setminus{\partial_1}} \cup \omega_\alpha \cup \omega^1_{\partial_1}$$

\textbf{Равенство для второго класса:}

$$\partial_2 \defeq \partial\dfrac{\omega^\sigma_{\sigma\setminus\rho}}{\omega^\Delta_{\sigma\setminus\rho}}(\alpha) = \partial\dfrac{\omega^\sigma_{\sigma\setminus\rho}}{\omega^\Delta_{\Delta\setminus\alpha}}(\alpha)$$
$$gen(\varnothing, \Delta) =^2 \omega^\sigma_{\sigma\setminus\rho\setminus{\partial_1}} \cup \omega_{\alpha\setminus\partial_2} \cup \omega^1_{\partial_1} \cup gen(\omega^\sigma_{\sigma\setminus\rho\setminus{\partial_1}} \cup \omega_{\alpha\setminus\partial_2} \cup \omega^1_{\partial_1}, \partial_2) \eqdef$$
$$\eqdef \omega^\sigma_{\sigma\setminus\rho\setminus{\partial_1}} \cup \omega_{\alpha\setminus\partial_2} \cup \omega^1_{\partial_1} \cup \omega^2_{\partial_2}$$

\textbf{Равенство для третьего класса:}

$$\partial_3 \defeq \partial\dfrac{\omega^\sigma_{\sigma\setminus\rho\setminus{\partial_1}} \cup \omega_\alpha}{\omega^\Delta_{\Delta\setminus\partial_1}}(\partial_1)$$
$$gen(\varnothing, \Delta) =^3 \omega^\sigma_{\sigma\setminus\rho\setminus{\partial_1}} \cup \omega_{\alpha\setminus\partial_2} \cup \omega^1_{\partial_1\setminus\partial_3} \cup \omega^2_{\partial_2} \cup gen(\omega^\sigma_{\sigma\setminus\rho\setminus{\partial_1}} \cup \omega_{\alpha\setminus\partial_2} \cup \omega^1_{\partial_1\setminus\partial_3} \cup \omega^2_{\partial_2}, \partial_3) \eqdef$$
$$\eqdef \omega^\sigma_{\sigma\setminus\rho\setminus{\partial_1}} \cup \omega_{\alpha\setminus\partial_2} \cup \omega^1_{\partial_1\setminus\partial_3} \cup \omega^2_{\partial_2} \cup \omega^3_{\partial_3}$$

И так далее: чтобы получить выражение для равенства с точностью до $k$-го класса, берём $$\partial_k \defeq \partial\dfrac{\text{тут контекст для }\partial_{k-2}}{\omega^\Delta_{\Delta\setminus\partial_{k-2}}}(\partial_{k-2})$$ вычитаем его из порождения $\omega^{k-2}_{\partial_{k-2}}$ в выражении для равенства с точностью до $(k-1)$-го класса, используем это как контекст для $\omega^k_{\partial_k} = gen(..., \partial_k)$ и складываем.

\end{document}
