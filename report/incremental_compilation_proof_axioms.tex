\documentclass[a4paper,12pt]{report} %%%{article}

\usepackage{cmap} % searchable PDFs
\usepackage[T2A]{fontenc} % scalable fonts
\usepackage[utf8]{inputenc} % input in UTF8
%\usepackage[english,russian]{babel} % dashes on linebreaks
\usepackage{indentfirst} % indents in paragraphs
\usepackage{amstext,amssymb,amsfonts,amsmath,mathtext,enumerate,float}
\usepackage[left=25mm,right=2cm,top=2cm,bottom=2cm,bindingoffset=0cm]{geometry}
\usepackage[unicode]{hyperref}
\usepackage{graphicx}
\usepackage{ulem} % strikethrough
\usepackage{verbatim} % multiline comments
\usepackage{hhline}

\usepackage{listings}
\usepackage{color}
 
\definecolor{dkgreen}{rgb}{0,0.6,0}
\definecolor{gray}{rgb}{0.5,0.5,0.5}
\definecolor{mauve}{rgb}{0.58,0,0.82}
 
\lstset{ %
  columns=flexible,
%  language=C,                     % the language of the code
  basicstyle=\footnotesize\ttfamily,       % the size of the fonts that are used for the code
  numbers=left,                   % where to put the line-numbers
  numberstyle=\tiny\color{gray},  % the style that is used for the line-numbers
  stepnumber=1,                   % the step between two line-numbers. If it's 1, each line 
                                  % will be numbered
  numbersep=5pt,                  % how far the line-numbers are from the code
  backgroundcolor=\color{white},  % choose the background color. You must add \usepackage{color}
  showspaces=false,               % show spaces adding particular underscores
  showstringspaces=false,         % underline spaces within strings
  showtabs=false,                 % show tabs within strings adding particular underscores
%  frame=single,                   % adds a frame around the code
  rulecolor=\color{black},        % if not set, the frame-color may be changed on line-breaks within not-black text (e.g. comments (green here))
  tabsize=4,                      % sets default tabsize
  captionpos=b,                   % sets the caption-position to bottom
  breaklines=true,                % sets automatic line breaking
  breakatwhitespace=false,        % sets if automatic breaks should only happen at whitespace
  title=\lstname,                 % show the filename of files included with \lstinputlisting;
                                  % also try caption instead of title
  keywordstyle=\color{blue},      % keyword style
  commentstyle=\color{dkgreen},   % comment style
  stringstyle=\color{mauve},      % string literal style
%  escapeinside={\%*}{*)},         % if you want to add LaTeX within your code
  mathescape=true,
  morekeywords={*,...},           % if you want to add more keywords to the set
  deletekeywords={...}            % if you want to delete keywords from the given language
}

\lstset{
literate={а}{{\selectfont\char224}}1
{б}{{\selectfont\char225}}1
{в}{{\selectfont\char226}}1
{г}{{\selectfont\char227}}1
{д}{{\selectfont\char228}}1
{е}{{\selectfont\char229}}1
{ё}{{\"e}}1
{ж}{{\selectfont\char230}}1
{з}{{\selectfont\char231}}1
{и}{{\selectfont\char232}}1
{й}{{\selectfont\char233}}1
{к}{{\selectfont\char234}}1
{л}{{\selectfont\char235}}1
{м}{{\selectfont\char236}}1
{н}{{\selectfont\char237}}1
{о}{{\selectfont\char238}}1
{п}{{\selectfont\char239}}1
{р}{{\selectfont\char240}}1
{с}{{\selectfont\char241}}1
{т}{{\selectfont\char242}}1
{у}{{\selectfont\char243}}1
{ф}{{\selectfont\char244}}1
{х}{{\selectfont\char245}}1
{ц}{{\selectfont\char246}}1
{ч}{{\selectfont\char247}}1
{ш}{{\selectfont\char248}}1
{щ}{{\selectfont\char249}}1
{ъ}{{\selectfont\char250}}1
{ы}{{\selectfont\char251}}1
{ь}{{\selectfont\char252}}1
{э}{{\selectfont\char253}}1
{ю}{{\selectfont\char254}}1
{я}{{\selectfont\char255}}1
{А}{{\selectfont\char192}}1
{Б}{{\selectfont\char193}}1
{В}{{\selectfont\char194}}1
{Г}{{\selectfont\char195}}1
{Д}{{\selectfont\char196}}1
{Е}{{\selectfont\char197}}1
{Ё}{{\"E}}1
{Ж}{{\selectfont\char198}}1
{З}{{\selectfont\char199}}1
{И}{{\selectfont\char200}}1
{Й}{{\selectfont\char201}}1
{К}{{\selectfont\char202}}1
{Л}{{\selectfont\char203}}1
{М}{{\selectfont\char204}}1
{Н}{{\selectfont\char205}}1
{О}{{\selectfont\char206}}1
{П}{{\selectfont\char207}}1
{Р}{{\selectfont\char208}}1
{С}{{\selectfont\char209}}1
{Т}{{\selectfont\char210}}1
{У}{{\selectfont\char211}}1
{Ф}{{\selectfont\char212}}1
{Х}{{\selectfont\char213}}1
{Ц}{{\selectfont\char214}}1
{Ч}{{\selectfont\char215}}1
{Ш}{{\selectfont\char216}}1
{Щ}{{\selectfont\char217}}1
{Ъ}{{\selectfont\char218}}1
{Ы}{{\selectfont\char219}}1
{Ь}{{\selectfont\char220}}1
{Э}{{\selectfont\char221}}1
{Ю}{{\selectfont\char222}}1
{Я}{{\selectfont\char223}}1
}
\usepackage{mathtools}
\newcommand{\defeq}{\vcentcolon=}
\newcommand{\eqdef}{=\vcentcolon}

\renewcommand{\contentsname}{Содержание} 
\setcounter{secnumdepth}{0}
\setcounter{tocdepth}{3}

%\sloppy % align=justify

\begin{document}

\section{Основные определения}

$\Sigma$ --- множество входов, $\Omega$ --- множество выходов (н.б.ч.с.). Функция порождения
выходов по входам (частичная):

$$
gen : 2^\Omega\to 2^\Sigma\to 2^\Omega
$$

Аксиомы функции порождения:

\begin{itemize}
	\item если $gen(\omega,\sigma)$ --- определена, то $\omega\cap gen(\omega,\sigma)=\varnothing$;
	\item если $gen(\omega,\sigma)=\omega^\prime$, то существует единственное дизъюнктное разбиение $\omega^\prime=\bigcup^\varnothing_{s\in\sigma}\omega^\prime_s$, 
	удовлетворяющее свойству 

	$$\forall s\in\sigma : gen(\omega\cup\omega^\prime\setminus\omega^\prime_s,\{s\})=\omega^\prime_s$$

	\item $\forall s\in\Sigma,\; \forall\omega,\:\omega^\prime\subseteq\Omega:$ если $gen(\omega,\{s\})$ и $gen(\omega^\prime,\{s\})$ ---
	определены, то $gen(\omega,\{s\})=gen(\omega^\prime,\{s\})$;

	\item если $gen(\omega,\{s\})$ --- определена, то в $\omega$ существует наименьшее по включению подмножество $d_\omega(s)$, такое, что
	$gen(d_\omega(s), \{s\})$ --- определена;

	\item если $gen(\omega,\sigma)=\omega^\prime$ и для какого-то $s\in\sigma$ существует $s_1\in\sigma$, такой, что
	% $d_\omega(s)\cap \omega^\prime_{s_1}\ne\varnothing$, 
	$d_{\omega\cup\omega^\prime\setminus\omega^\prime_s}(s) \cap \omega^\prime_{s_1}\ne\varnothing$, 
	то $gen(\omega,\sigma\setminus s_1)$ --- не определена;

	\item если $gen(\omega,\sigma_1)\cap gen(\omega,\sigma_2)\ne\varnothing$, то $gen(\omega,\sigma_1\cup\sigma_2)$ --- не определена.
\end{itemize}

\hrulefill

Кажется, четвёртая аксиома несовместима с пятой: $d_\omega(s)$, о котором идёт речь в (5), существует (по (4)) в предположении, что определено $gen(\omega, \{s\})$; но по утверждению (5), $gen(\omega, \sigma\setminus s_1)$ может быть не определено; ясно, что в таком случае не может быть определено и $gen(\omega, \{s\})$. Предлагаю следующую формулировку (4):

\begin{itemize}
	\item если $gen(\omega,\sigma)=\omega^\prime$ --- определена, то $\forall s \in \sigma$ в $\omega \cup \omega^\prime \setminus \omega^\prime_s$ существует наименьшее по включению подмножество $d_{\omega,\omega^\prime}(s)$, такое, что $gen(d_{\omega,\omega^\prime}(s), \{s\})$ --- определена.
\end{itemize}

О шестой аксиоме: 1) без дополнительных условий она верна только для $\sigma_1\cap\sigma_2 = \varnothing$; 2) похоже, она выводится из остальных:

Пусть $gen(\omega,\sigma_1) = \omega_1$, $gen(\omega,\sigma_2) = \omega_2$, $\omega_1\cap\omega_2\ne\varnothing$. Предположим, что $gen(\omega,\sigma_1\cup\sigma_2)$ определена и равна $\omega^\prime$. Тогда по аксиоме о дизъюнктном разбиении можно (объединяя порождения для соответствующих $s$ и учитывая, что $\sigma_1\cap\sigma_2 = \varnothing$) разделить $\omega^\prime$ на $\omega^\prime_1$ и $\omega^\prime_2$, где $\omega^\prime_1 = gen(\omega\cup\omega^\prime_2,\sigma_1)$, $\omega^\prime_2 = gen(\omega\cup\omega^\prime_1,\sigma_2)$, $\omega^\prime_1 \cap \omega^\prime_2 = \varnothing$. Но по аксиоме ``определены-значит-равны'' $\omega_1 = \omega^\prime_1$ и $\omega_2 = \omega^\prime_2$; таким образом, $\omega_1$ и $\omega_2$ одновременно пересекаются и не пересекаются, значит, предположение об определённости $gen(\omega,\sigma_1\cup\sigma_2)$ неверно.
\\

Наконец, очевидное следствие из аксиомы о дизъюнктном разбиении:

\begin{itemize}
	\item если $gen(\omega,\sigma)=\omega^\prime$, то для любого $\tau \subseteq \sigma$ можно определить $\omega^\prime_\tau =\bigcup\limits_{s\in\tau}\omega^\prime_s$, при этом $\tau_1 \cap \tau_2 = \varnothing \Leftrightarrow \omega^\prime_{\tau_1} \cap \omega^\prime_{\tau_2} = \varnothing$.
\end{itemize}

\hrulefill

\textbf{Теорема 1.} 

Пусть дано: $\sigma \subset \Sigma$, $gen(\varnothing, \sigma) = \omega^\sigma$, $\forall s \in \sigma$ $d_{\varnothing,\omega}(s) \subset 2^\Omega$. Пусть $\rho, \alpha \subset \Sigma$, при этом $\rho \subseteq \sigma$, $\sigma \cap \alpha = \varnothing$; $\Delta = \Delta^\rho_\alpha\sigma = \sigma\setminus\rho\cup\alpha$, $gen(\varnothing, \Delta) = \omega^\Delta$. Определим $\xi = \{s \in \sigma\setminus\rho \mid d_{\varnothing,\omega}(s) \cap \omega_\rho \neq \varnothing\}$ (зависимые от $\rho$ элементы). 
Тогда:

$$gen(\varnothing, \Delta) = \omega^\sigma \setminus \omega^\sigma_\rho \setminus \omega^\sigma_\xi \cup gen(\omega^\sigma \setminus \omega^\sigma_\rho \setminus \omega^\sigma_\xi, \alpha \cup \xi)$$

\textbf{Доказательство:}

Сначала докажем равенство не для $\xi$, а для произвольного $\tau \subset \sigma\setminus\rho$, для которого определено $gen(\omega^\sigma \setminus \omega^\sigma_\rho \setminus \omega^\sigma_\tau, \alpha\cup\tau)$. Смысл этого такой, что равенство выполнится не только при перекомпиляции $\xi$, но и при перекомпиляции, например, множества, большего $\xi$.
\\

Слева направо: пусть $x \in gen(\varnothing, \Delta) = \omega^\Delta$. Докажем, что $x$ принадлежит правой части. По аксиоме о дизъюнктном разбиении существует единственное разбиение $\omega^\Delta = \bigcup^\varnothing_{s\in\Delta}\omega^\Delta_s$: $\forall s \in \Delta$ $\omega^\Delta_s = gen(\omega^\Delta\setminus\omega^\Delta_s, \{s\})$, следовательно, $\exists ! s_x \in \Delta$: $x \in \omega^\Delta_{s_x}$. Докажем теперь, что $gen(..., \{s_x\})$ для какого-то контекста определено и лежит в правой части; тогда по аксиоме ``определены-значит-равны'' $gen(..., \{s_x\}) = gen(\omega^\Delta\setminus\omega^\Delta_{s_x}, \{s_x\}) = \omega^\Delta_{s_x}$ и, следовательно, содержит $x$; значит, $x$ принадлежит правой части. Поскольку $s_x \in \Delta$, то рассмотрим два случая: a) $s_x \in \sigma\setminus\rho\setminus\tau$; b) $s_x \in \alpha\cup\tau$.

В случае (a): по аксиоме о дизъюнктном разбиении существует единственное разбиение для $gen(\varnothing, \sigma) = \omega^\sigma = \bigcup^\varnothing_{s\in\sigma}\omega^\sigma_s$: $\forall s \in \sigma$ $\omega^\sigma_s = gen(\omega^\sigma\setminus\omega^\sigma_s, \{s\})$. Поскольку $s_x \in \sigma$, то $\exists ! \omega^\sigma_{s_x} \subseteq \omega^\sigma$. Так как $s_x \notin \rho$, $s_x \notin \tau$, то $\omega^\sigma_{s_x} \cap \omega^\sigma_\rho = \varnothing$, $\omega^\sigma_{s_x} \cap \omega^\sigma_\tau = \varnothing$. Следовательно, $gen(\omega^\sigma\setminus\omega^\sigma_{s_x}, \{s_x\}) = \omega^\sigma_{s_x} \subseteq \omega^\sigma \setminus \omega^\sigma_\rho \setminus \omega^\sigma_\tau$, то есть определено и лежит в правой части. Для случая (a) включение левой части в правую доказано.

В случае (b): мы считали, что $gen(\omega^\sigma \setminus \omega^\sigma_\rho \setminus \omega^\sigma_\tau, \alpha\cup\tau)$ определено; пусть оно равно $\omega^{\alpha\cup\tau}$; по аксиоме о дизъюнктном разбиении определено и $gen((\omega^\sigma \setminus \omega^\sigma_\rho \setminus \omega^\sigma_\tau) \cup (\omega^{\alpha\cup\tau} \setminus \omega^{\alpha\cup\tau}_{s_x}), \{s_x\})$; кроме того, оно лежит в правой части как составная часть $gen(\omega \setminus \omega_\rho \setminus \omega_\tau, \alpha\cup\tau)$. Для случая (b) включение левой части в правую доказано.
\\

Справа налево: пусть $x$ принадлежит правой части; это означает, что либо a) $x \in \omega^\sigma \setminus \omega^\sigma_\rho \setminus \omega^\sigma_\tau$, либо b) $x \in gen(\omega^\sigma \setminus \omega^\sigma_\rho \setminus \omega^\sigma_\tau, \alpha \cup \tau)$.

В случае (a): рассмотрим $\omega^\sigma \setminus \omega^\sigma_\rho \setminus \omega^\sigma_\tau = \bigcup^\varnothing_{s \in \sigma} \omega^\sigma_s \setminus \bigcup^\varnothing_{s \in \rho} \omega^\sigma_s \setminus \bigcup^\varnothing_{s \in \tau} \omega^\sigma_s = \bigcup^\varnothing_{s \in \sigma\setminus\rho\setminus\tau} \omega^\sigma_s$ (равенства верны по аксиоме о дизъюнктном разбиении). Тогда ясно, что $\exists ! s_x \in \sigma\setminus\rho\setminus\tau$: $x \in \omega^\sigma_{s_x} = gen(\omega^\sigma \setminus \omega^\sigma_{s_x}, \{s_x\})$.

В случае (b): по аксиоме о дизъюнктном разбиении $\exists ! s_x \in \alpha\cup\tau$: $x \in \omega^{\alpha\cup\tau}_{s_x} = gen((\omega^\sigma \setminus \omega^\sigma_\rho \setminus \omega^\sigma_\tau) \cup (\omega^{\alpha\cup\tau} \setminus \omega^{\alpha\cup\tau}_{s_x}), \{s_x\})$.

В обоих случаях мы указали $gen(..., \{s_x\})$ для некоторого контекста, лежащее в правой части и содержащее $x$; также в обоих случаях $s_x \in \Delta$ и, следовательно, в левой части определено $\omega^\Delta_{s_x} = gen(\omega^\Delta\setminus\omega^\Delta_{s_x}, \{s_x\})$, лежащее в левой части. По аксиоме ``определены-значит-равны''  $\omega^\Delta_{s_x}$ равен $gen(..., \{s_x\})$ и, значит, также содержит $x$; следовательно, $x$ принадлежит левой части. Равенство для $\tau$ доказано.
\\

Теперь нужно доказать, что определено $gen(\omega^\sigma \setminus \omega^\sigma_\rho \setminus \omega^\sigma_\xi, \alpha\cup\xi)$. Проблема здесь в том, что среди наших аксиом нет критерия определённости $gen$, а есть только список ситуаций, когда она не определена. Можно считать этот список исчерпывающим, тогда для определённости $gen(..., \alpha\cup\xi)$ достаточно проверить, что $\alpha$ и $\xi$ не имеют общих порождений (аксиома 6), что порождения $\alpha$ и $\xi$ не пересекаются с $\omega^\sigma \setminus \omega^\sigma_\rho \setminus \omega^\sigma_\xi$ (аксиома 1), а также что $\omega^\sigma \setminus \omega^\sigma_\rho \setminus \omega^\sigma_\xi$ действительно содержит все необходимые зависимости для $\alpha$ и $\xi$ с учётом взаимозависимостей (аксиома 5). Тогда получим ЧТД.

\begin{comment}
	Можно ввести ещё одну аксиому: значение $gen(\omega, \sigma)$ не определено, если a) $\exists \sigma_1, \sigma_2 \subset \sigma$: $\sigma_1 \cap \sigma_2 = \varnothing$, $\exists \omega_1, \omega_2 \in 2^\Omega$: $gen(\omega_1, s_1) \cap gen(\omega_2, s_2) \neq \varnothing$, или b) $\exists \sigma^\prime \subset \sigma$, $\exists \omega^\prime \in 2^\Omega$: $gen(\omega^\prime, \sigma^\prime) \cap \omega \neq \varnothing$, или c) $\exists s \in \sigma$, $\exists \omega^\prime,\omega^{\prime\prime} \in 2^\Omega$: $d_{\omega^\prime,\omega^{\prime\prime}}(s) \nsubseteq \omega$; иначе оно определено.
\end{comment}

Короткое: понятно, что равенство имеет смысл только в случае определённости $gen(\varnothing, \Delta)$ и $gen(\varnothing, \sigma)$. По второй аксиоме ясно, что левая часть раскладывается на два множества $\omega^\Delta_{\sigma\setminus\rho\setminus\xi}$ и $\omega^\Delta_{\xi\cup\alpha}$. Первое из них по третьей аксиоме равно первому члену объединения из правой части (определённость которого следует из определённости $gen(\varnothing, \sigma))$. Второе из них равно второму члену объединения из правой части потому, что они являются результатами двух значений gen с одним и тем же множеством в качестве входа и равными контекстами (контексты --- это соответствующие первые члены частей). Более строго: 

Пусть $gen(\varnothing, \Delta)$ и $gen(\varnothing, \sigma)$ определены. По аксиоме о дизъюнктном разбиении определены и $\omega^\Delta_{\sigma \setminus \rho \setminus \xi}$ и $\omega^\sigma_{\sigma \setminus \rho \setminus \xi}$; и по аксиоме (3) $\omega^\Delta_{\sigma \setminus \rho \setminus \xi} = \omega^\sigma_{\sigma \setminus \rho \setminus \xi}$. Поскольку $\Delta = (\sigma \setminus \rho \setminus \xi) \cup (\xi \cup \alpha)$, то $gen(\varnothing, \Delta) = \omega^\Delta = \omega^\Delta_{\sigma \setminus \rho \setminus \xi} \cup \omega^\Delta_{\xi \cup \alpha}$. Но $\omega^\Delta_{\xi \cup \alpha} = gen(\omega^\Delta \setminus \omega^\Delta_{\xi \cup \alpha}, \xi \cup \alpha) = gen(\omega^\Delta_{\sigma \setminus \rho \setminus \xi}, \xi \cup \alpha) = gen(\omega^\sigma_{\sigma \setminus \rho \setminus \xi}, \xi \cup \alpha) = gen(\omega^\sigma \setminus \omega^\sigma_\rho \setminus \omega^\sigma_\xi, \alpha\cup\xi)$ из правой части.
\\

1. Пусть $\omega$, $\tilde{\omega}$ --- множества выходов, $\sigma$ --- множество входов.
	Тогда $\partial\frac{\omega}{\tilde{\omega}}\sigma$ --- это наименьшее подмножество $\sigma$, удовлетворяющее свойству $gen(\omega,\sigma\setminus\partial\frac{\omega}{\tilde{\omega}}\sigma)$ определено тогда и только тогда, когда $gen(\tilde{\omega},\sigma\setminus\partial\frac{\omega}{\tilde{\omega}}\sigma)$ определено.
\\

2. Свойство: $\partial$ всегда определен и (на худой конец) равен $\sigma$.
\\

3. Свойство: $\partial\frac{\omega}{\omega}\sigma=\varnothing$.
\\

Вопрос: верно ли, что в условиях инкрементального случая (там, где $\Delta^\rho_\alpha\sigma$)

$$\partial\frac{\omega}{\omega\setminus\omega_\rho\cup\omega_\alpha}(\sigma\setminus\rho)\subseteq\xi$$ ? 

\newpage

\end{document}