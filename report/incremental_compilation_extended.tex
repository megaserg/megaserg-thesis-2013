\documentclass[a4paper,12pt]{report} %%%{article}

\usepackage{cmap} % searchable PDFs
\usepackage[T2A]{fontenc} % scalable fonts
\usepackage[utf8]{inputenc} % input in UTF8
\usepackage[english,russian]{babel} % dashes on linebreaks
\usepackage{indentfirst} % indents in paragraphs
\usepackage{amstext,amssymb,amsfonts,amsmath,mathtext,enumerate,float}
\usepackage[left=25mm,right=2cm,top=2cm,bottom=2cm,bindingoffset=0cm]{geometry}
\usepackage[unicode]{hyperref}
\usepackage{graphicx}
\usepackage{ulem} % strikethrough
\usepackage{verbatim} % multiline comments
\usepackage{hhline}

\usepackage{listings}
\usepackage{color}
 
\lstset{extendedchars=\true}
  
\definecolor{dkgreen}{rgb}{0,0.6,0}
\definecolor{gray}{rgb}{0.5,0.5,0.5}
\definecolor{mauve}{rgb}{0.58,0,0.82}
 
\lstset{ %
  columns=flexible,
%  language=C,                     % the language of the code
  basicstyle=\footnotesize\ttfamily,       % the size of the fonts that are used for the code
  numbers=left,                   % where to put the line-numbers
  numberstyle=\tiny\color{gray},  % the style that is used for the line-numbers
  stepnumber=1,                   % the step between two line-numbers. If it's 1, each line will be numbered
  numbersep=5pt,                  % how far the line-numbers are from the code
  backgroundcolor=\color{white},  % choose the background color. You must add \usepackage{color}
  showspaces=false,               % show spaces adding particular underscores
  showstringspaces=false,         % underline spaces within strings
  showtabs=false,                 % show tabs within strings adding particular underscores
%  frame=single,                   % adds a frame around the code
  rulecolor=\color{black},        % if not set, the frame-color may be changed on line-breaks within not-black text (e.g. comments (green here))
  tabsize=4,                      % sets default tabsize
  captionpos=b,                   % sets the caption-position to bottom
  breaklines=true,                % sets automatic line breaking
  breakatwhitespace=false,        % sets if automatic breaks should only happen at whitespace
  title=\lstname,                 % show the filename of files included with \lstinputlisting; also try caption instead of title
  keywordstyle=\color{blue},      % keyword style
  commentstyle=\color{dkgreen},   % comment style
  stringstyle=\color{mauve},      % string literal style
%  escapeinside={\%*}{*)},         % if you want to add LaTeX within your code
  mathescape=true,
  morekeywords={*,...},           % if you want to add more keywords to the set
  deletekeywords={...},           % if you want to delete keywords from the given language
  aboveskip=1em,
  belowskip=-1em
}


\usepackage{mathtools}
\newcommand{\defeq}{\vcentcolon=}
\newcommand{\eqdef}{=\vcentcolon}

\renewcommand{\contentsname}{Содержание} 
\setcounter{secnumdepth}{0}
\setcounter{tocdepth}{3}

% \linespread{1.3}
\sloppy % align=justify


\begin{document}

\newcommand{\Deltasigma}{\Delta_\alpha^\rho\sigma}

Пусть дано: $\sigma$, $\forall s \in \sigma$ $d(s) \subset 2^\Omega$, $\omega = gen(\varnothing, \sigma)$. Предположим, что $\forall \sigma' \subseteq \sigma$ также известно $\omega_{\sigma'}$~--- множество порождений, которые были сгенерированы из элементов $\sigma'$ во время вычисления $gen(\varnothing, \sigma)$ (можем ли такое предположить? я написал бы $\omega_{\sigma'} = gen(\omega_{\sigma\setminus\sigma'}, \sigma')$, где $\omega_{\sigma\setminus\sigma'} = gen(\omega_{\sigma'}, \sigma\setminus\sigma')$, но меня смущает эта непрямая рекурсивность, к тому же это не совсем то --- например, если $\sigma'=\{s_1\}$ и $s_1$ взаимозависим с $s_2$ и больше они ни от чего не зависят, то $\omega_{\sigma'}$ при таком определении получится пустым, хотя не должно).\\

Пусть есть $\Deltasigma = \sigma\setminus\rho\cup\alpha$. Хотим понять, как посчитать $gen(\varnothing, \Deltasigma)$ на основе $gen(\varnothing, \sigma)$. Определим $\xi = \{s \in \sigma\setminus\rho \mid d(s) \cap \omega_\rho \neq \varnothing\}$ (зависимые от $\rho$ элементы). Обозначим также $A = gen(\omega_{\sigma\setminus\rho}, \alpha)$. Также возможно, что в $\alpha$ содержатся необходимые зависимости для каких-то элементов из $\sigma$: пусть $\chi = \{s \in \sigma\setminus\rho \mid \omega_{\{s\}} = \varnothing\}$. Тогда
$$gen(\varnothing, \Deltasigma) = \omega \setminus \omega_\rho \setminus \omega_\xi \cup A \cup gen(\omega \setminus \omega_\rho \setminus \omega_\xi \cup A, \xi) \cup gen(A \cup \omega_{\sigma\setminus\rho}, \chi)$$

Теперь возникает такая ситуация: что если в последнем члене --- $gen(A \cup \omega_{\sigma\setminus\rho}, \chi)$ --- есть какие-то порождения, которых не хватало при вызове генерации для $\alpha$ или $\xi$? Пусть $\chi'$ --- те входы из $\alpha$ и $\xi$, для которых из-за этого не сгенерировалось никаких порождений. Нужно вызвать генерацию для них. Это означает, что могли удовлетвориться зависимости ещё каких-то элементов из $\sigma\setminus\rho$, которые теперь нужно перекомпилировать и так далее. Видимо, здесь и возникает итеративность?

\end{document}