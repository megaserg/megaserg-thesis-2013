\newpage
\tableofcontents

\chapter{Формальная модель}

Пусть $\Sigma$~--- множество входов, $\Omega$~--- множество всех возможных порождений. Выделим специальное множество $ERROR \subset \Omega$\footnote{Здесь и далее $\subset$ означает ``является собственным подмножеством'', а $\subseteq$~--- ``равно или является собственным подмножеством''.}.\\
Для $\sigma \subset \Sigma$, $s \in \Sigma$ определим $gen_\sigma(s) \subset \Omega$~--- порождения $s$ в контексте $\sigma$. Таким образом, тип функции $gen: 2^\Sigma \times \Sigma \to 2^\Omega$.\\
Контекст $\sigma$ назовём \textit{необходимым} для входа $s$, если для любого множества $\varphi: \varphi \subset \sigma$ $gen_\varphi(s) \subseteq ERROR$ или $gen_\varphi(s) = \varnothing$.\\
Контекст $\sigma$ назовём \textit{достаточным} для входа $s$, если для любого множества $\Phi: \sigma \subseteq \Phi$ $gen_\Phi(s) = gen_\sigma(s)$.\\
Доопределим $gen$ для набора $S \subset \Sigma$: $gen_\sigma(S) \defeq \bigcup\limits_{s \in S} gen_\sigma(s)$.\\
В частности, $gen_\sigma(\sigma) = \bigcup\limits_{s \in \sigma} gen_\sigma(s)$~--- порождение набора входов в контексте этого же набора.\\
Пусть для каждого набора $\sigma \subset \Sigma$, входа $s \in \sigma$ задана функция $d(\sigma, s) \subseteq \sigma$~--- некоторый необходимый и достаточный контекст для входа $s$. TODO: что если таких несколько?\\
Тогда $gen^d(\sigma) \defeq \bigcup\limits_{s \in \sigma} gen_{d(\sigma, s)}(s)$.

\subsection{Задача инкрементальной компиляции}

\newcommand{\Deltasigma}{\Delta_\alpha^\rho\sigma}
Рассмотрим $\Deltasigma = (\sigma \setminus \rho) \cup \alpha$, где $\rho, \alpha \subset \Sigma$, при этом $\rho \subseteq \sigma$, $\sigma \cap \alpha = \varnothing$. Тогда
$$gen^d(\Deltasigma) \overset{def}{=} \bigcup\limits_{s \in \Deltasigma} gen_{d(\Deltasigma, s)}(s) = \bigcup\limits_{s \in (\sigma \setminus \rho) \cup \alpha} gen_{d((\sigma \setminus \rho) \cup \alpha, s)}(s).$$
Требуется выразить эту функцию через $gen^d(\sigma)$.

\subsection{Задача переиспользования порождений других входов}

Пусть есть $\sigma_1, \dots, \sigma_n$ ($\sigma_i \subset \Sigma$)~--- наборы входов и известны $\omega_1, \dots, \omega_n$, где $\omega_i = gen^d(\sigma_i)$.\\
Также есть набор $\sigma^*$, при этом $\sigma^* \subset \bigcup\limits_{i \in [1:n]}\sigma_i$, т.е. $\forall s \in \sigma^*$ $\exists i_s \in [1:n]: s \in \sigma_{i_s}$.\\
Требуется получить $gen^d(\sigma^*)$, используя известные порождения $\omega_1, \dots, \omega_n$.

\section{Случай компиляции}

Пусть у нас есть основной репозиторий, в котором лежат исходные файлы. Содержимое основного репозитория меняется от ревизии к ревизии. Также есть несколько локальных копий репозитория, каждая из которых была синхронизирована с репозиторием в какой-то момент времени и, возможно, имеет какие-то локальные изменения.\\

Параметризуем наши сущности моментом времени $t$, но будем опускать его при записи. Пусть $n = n^t$~--- количество файлов в основном репозитории в момент  $t$, $s = s^t$~--- количество ревизий/версий в основном репозитории в момент $t$ (можно считать, что они упорядочены по времени создания, т.е. номер ревизии возрастает со временем), $r = r^t$~--- количество локальных копий основного репозитория в момент $t$.\\

Будем рассматривать два случая: 1) когда для каждого файла история его изменений хранится отдельно, но нумерация ревизий всё равно сквозная (будем называть его ``случай SVN'') и 2) когда каждая ревизия~--- снимок всего репозитория (будем называть его ``случай Git''). В первом случае для каждого файла в любой момент известен номер последней ревизии, в которой он изменялся; во втором случае каждый файл ``обновляется'' в каждой ревизии, но это изменение ненулевое только тогда, когда файл изменялся фактически.\\

Обозначим $SRC \subset \{0,1\}^*$~--- множество всех возможных исходных файлов, $CF \subset \{0,1\}^*$~--- множество всех возможных класс-файлов, $ERR$~--- множество всех возможных ошибок компиляции.
Пусть $FN = FN^t = \{FN_1, \dots, FN_n\}$~--- пути к файлам в репозитории в момент $t$ (или можно дополнительно ввести набор директорий, понятия вложенности директорий и принадлежности файла директории~--- тогда $FN_i$ будет не путём к файлу, а его именем).\\

Введём обозначения для информации, хранящейся для файлов в основном репозитории.\\
В случае SVN:\\
$\forall i \in [1:n]$\\
\indent a) $cont(i) = cont^t(i) \in SRC$~--- содержимое файла $i$ в момент $t$;\\
\indent b) $ts(i) = ts^t(i)$~--- timestamp последнего изменения файла $i$ в момент $t$;\\
\indent c) $lv(i) = lv^t(i) \in [1:s]$~--- последняя на момент $t$ ревизия, когда изменялся файл $i$.\\
В случае Git:\\
$\forall v \in [1:s], \forall i \in [1:n]$\\
\indent a) $cont_v(i) \in SRC$~--- содержимое файла $i$ в ревизии $v$;\\
\indent b) $ts_v(i)$~--- timestamp последнего изменения файла $i$ в ревизии $v$.\\

Похожим образом обозначим информацию, хранящуюся для файлов в локальных копиях.\\
Пусть $R = R^t = \{R_1, \dots, R_r\}$~--- локальные копии репозитория.\\
$\forall l \in [1:r], \forall i \in [1:n]$\\
\indent a) $loccont_l(i) = loccont_l^t(i) \in SRC$~--- содержимое файла $i$ в локальной копии $l$ в момент $t$;\\
\indent b) $locts_l(i) = locts_l^t(i)$~--- timestamp последнего изменения файла $i$ в локальной копии $l$ в момент $t$;\\
\indent c) $loclv_l(i) = loclv_l^t(i) \in {[1:s]}$~--- последняя на момент $t$ ревизия основного репозитория, с которой был синхронизирован файл $i$ в локальной копии $l$;\\
(В случае Git $\forall i \in [1:n]$ $loclv_l(i)$ одинаковы~--- это последняя на момент $t$ ревизия основного репозитория, с которой была синхронизирована локальная копия $l$.)\\
\indent d) $lochash_l(i) = lochash_l^t(i) = hash(loccont_l^t(i)) $~--- хэш содержимого файла $i$ в локальной копии $l$ в момент $t$.

Таким образом, локальное изменение файла $i$ в копии $l$ (в момент $t$): $$\mathit{diff}(loccont_l(i), cont_{loclv_l(i)}(i)),$$ где $\mathit{diff}$~--- функция вычисления разности двух исходных файлов.\\
TODO: случай несуществующего файла.

\subsection{Зависимости}

Пусть $V = \{v_1, \dots, v_n\}$~--- набор файлов в состоянии некоторого репозитория (основного или локальной копии) в некоторый момент времени, где каждое $v_i$ является парой $(path_i, cont_i)$, которая означает, что в файле по пути $path_i$ в этом репозитории в этот момент времени записано содержимое $cont_i \in SRC$. Пусть мы умеем считать зависимости для файлов; тогда можно построить набор зависимостей $E(V) = \{e_1, \dots, e_m\}$, где каждое $e_j$ является тройкой $(\textit{type}_j, \textit{src}_j, \textit{dst}_j)$, которая означает, что файл $\textit{src}_j$ зависит от файла $\textit{dst}_j$, при этом тип зависимости $\textit{type}_j$.\\

Тогда рассмотрим $\Sigma = V, \Omega = CF \cup ERR$. $\forall \sigma \subseteq V$, $\forall s \in V$ можем задать функцию $gen_\sigma(s) = \textit{compile}(\sigma, s) \subset CF \cup ERR$~--- результат компиляции файла $s$ при условии, что рядом в SOURCEPATH лежат файлы $\sigma$. Можем определить функцию $d(V,s) \subseteq V$~--- набор файлов из $V$, необходимый для безошибочной компиляции файла $s$ и при этом достаточный в том смысле, что от добавления в него других файлов результат компиляции не меняется. (TODO: верно ли, что при фиксированных $V$ и $s \in V$ $\forall \sigma \subseteq V$ либо $\textit{compile}(\sigma, s) \subset ERR$, либо $\textit{compile}(\sigma, s) = FILES_s \subset CF$, то есть для любого контекста компиляция выдаёт либо один и тот же результат, либо ошибку?) Эта функция может быть определена как транзитивное замыкание функции $dependencies$, возвращающей непосредственные зависимости и определяемой как $dependencies(V,s) \defeq \{v \mid \exists e_j \in E(V): src_j = s, dst_j = v\}$. (TODO: зависимости надо считать с учётом контекста?)

\subsection{Локальный кэш на timestamp}

Пусть $ct_l^t$~--- момент, когда в локальной копии $l$ была произведена компиляция, являющаяся последней на момент $t$. Например, пусть компиляция была произведена в некоторый момент $x$, тогда $ct_l^t$ будет равно $x$ для всех $t \geqslant x$, пока не будет произведена следующая компиляция.\\
$\forall l \in [1:r], \forall i \in [1:n]$\\
\indent a) $loclct_l^t(i) = locts_l^{ct_l^t}(i)$~--- timestamp последнего на момент $ct_l^t$ изменения файла $i$ в локальной копии $l$;\\
\indent b) $loccache_l^t(i) = compile(..., loccont_l^{ct_l^t}(i))$~--- результат последней компиляции файла $i$ в локальной копии $l$, то есть компиляции его содержимого на момент $ct_l^t$.\\

\subsection{Локальный кэш на хэшах}

Рассмотрим теперь локальный кэш на хэшах~--- некоторое хранилище, ассоциированное с локальной копией, в каждый момент времени содержащее результаты некоторых последних компиляций. Опять, пусть $ct_l^t$~--- момент, когда в локальной копии $l$ была произведена компиляция, являющаяся последней на момент $t$.
$\forall l \in [1:r], \forall i \in [1:n]$\\
\indent a) $loclh_l^t(i) = lochash_l^{ct_l^t}(i)$~--- хэш содержимого файла $i$ в локальной копии $l$ на момент последней компиляции, то есть $ct_l^t$;\\
\indent b) $loccache_l^t(i) = compile(..., loccont_l^{ct_l^t}(i))$~--- результат последней компиляции файла $i$ в локальной копии $l$, то есть компиляции его содержимого на момент $ct_l^t$.\\

Пусть есть некоторая локальная копия $l$. Тогда можно описать, что происходит с переменными, описывающими содержащуюся в ней информацию, при некоторых процессах.\\

\subsubsection[Функция \texttt{rebuild}]{Функция \texttt{rebuild}: полная компиляция}
\begin{lstlisting}
function $\mathbf{rebuild}(l)$:
	for $\forall i \in [1:n]$ do
		$loclct_l(i) \defeq locts_l(i)$
		$loclh_l(i) \defeq lochash_l(i)$
		$loccache_l(i) \defeq compile(..., loccont_l(i))$
		$result[i] \defeq loccache_l(i)$
	return $result$
\end{lstlisting}

\subsubsection[Функция \texttt{make\_ts}]{Функция \texttt{make\_ts}: инкрементальная компиляция с использованием локального кэша на timestamp}
\begin{lstlisting}
function $\mathbf{make\_ts}(l)$:
	$tocompile \defeq \varnothing$
	$result \defeq \varnothing$
	for $\forall i \in [1:n]$ do
		$result[i] \defeq loccache_l(i)$
	for $\forall i \in [1:n]$ do
		if $locts_l(i) \neq loclct_l(i)$ then
			$tocompile \defeq tocompile \cup \{i\} \cup dependents(i)$
	for $\forall i \in tocompile$ do
		$loclct_l(i) \defeq locts_l(i)$
		$loccache_l(i) \defeq compile(..., loccont_l(i))$
		$result[i] \defeq loccache_l(i)$
	return $result$
\end{lstlisting}

\subsubsection[Функция \texttt{make\_hash}]{Функция \texttt{make\_hash}: инкрементальная компиляция с использованием локального кэша на хэшах}
\begin{lstlisting}
function $\mathbf{make\_hash}(l)$:
	$tocompile \defeq \varnothing$
	$result \defeq \varnothing$
	for $\forall i \in [1:n]$ do
		$result[i] \defeq loccache_l(i)$
	for $\forall i \in [1:n]$ do
		if $lochash_l(i) \neq loclh_l(i)$ then
			$tocompile \defeq tocompile \cup \{i\} \cup dependents(i)$
	for $\forall i \in tocompile$ do
		$loclh_l(i) \defeq lochash_l(i)$
		$loccache_l(i) \defeq compile(..., loccont_l(i))$
		$result[i] \defeq loccache_l(i)$
	return $result$
\end{lstlisting}

\subsection{Глобальный кэш на хэшах}

Рассмотрим теперь глобальный, или удалённый, кэш на хэшах~--- некоторое хранилище, содержащее результаты компиляций для каждой ревизии в центральном репозитории.
$\forall v \in [1:s], \forall i \in [1:n]$\\
\indent a) $cont_v(i) \in SRC$~--- содержимое файла $i$ в ревизии $v$;\\
\indent b) $hash_v(i)$~--- хэш содержимого файла $i$ в ревизии $v$;\\
\indent c) $cache_v(i) = compile(..., cont_v(i))$~--- результат компиляции файла $i$ в ревизии $v$.\\

\subsubsection[Функция \texttt{make\_glob\_hash}]{Функция \texttt{make\_glob\_hash}: инкрементальная компиляция с использованием локального и глобального кэшей на хэшах}
Пусть есть локальная копия репозитория, локальный кэш на хэшах и глобальный (удалённый) кэш на хэшах. Рассмотрим ситуацию отсутствия зависимостей, тогда для каждого файла можно независимо считать хэш и порождения.
Зафиксируем некоторый файл. Пусть $hash$~--- хэш текущей локальной копии файла, $h1$~--- хэш файла на момент последнего вызова функции порождения, $c1$~--- результат последнего вызова функции порождения, $h2$~--- хэш в удалённом кэше, $c2$~--- результат компиляции, хранящийся в удалённом кэше.

\begin{lstlisting}
// $\text{Вызов функции порождения на текущей локальной копии (и обновление локального кэша).}$
function $\mathbf{regenerate}$:
	$h1 \defeq hash$
	$c1 \defeq gen(src)$
\end{lstlisting}
\begin{lstlisting}
// $\text{Обновление удалённого кэша.}$
function $\mathbf{upload}$:
	$h2 \defeq h1$
	$c2 \defeq c1$
\end{lstlisting}
\begin{lstlisting}
// $\text{Копирование удалённого кэша в локальный.}$
function $\mathbf{download}$:
	$h1 \defeq h2$
	$c1 \defeq c2$
\end{lstlisting}

Тогда можем описать функцию, которая использует кэши и не вызывает функцию порождения, если это не необходимо.

\begin{lstlisting}
// $\text{Версия 1}$
function $\mathbf{make\_glob\_hash}$:
	if $\exists h2$ then
		if $\exists h1$ then
			if $hash = h1$ then
				if $hash = h2$ then
					// nothing to do
				else
					upload
			else
				if $hash = h2$ then
					download (or regenerate)
				else
					regenerate
					upload
		else
			if $hash = h2$ then
				download (or regenerate)
			else
				regenerate
				upload
	else
		if $\exists h1$ then
			if $hash = h1$ then
				upload
			else
				regenerate
				upload
		else
			regenerate
			upload
\end{lstlisting}

Её можно переписать в более коротком виде:
\begin{lstlisting}
// $\text{Версия 2}$
function $\mathbf{make\_glob\_hash}$:
	if $\exists h2 \wedge hash = h2$ then
		if $\nexists h1 \vee hash \neq h1$ then
			download (or regenerate)
	else
		if $\nexists h1 \vee hash \neq h1$ then
			regenerate
		upload
\end{lstlisting}

\subsection{TODO}
Далее можно описать процедуры \texttt{commit} и \texttt{checkout}, процедуры добавления, удаления и изменения файлов.