\section{Обзор литературы}

В работе \cite{safeness_niels} анализируется проблема безопасности использования make для инкрементальной компиляции как альтернативы полной сборке (brute-force build). Безопасность инкрементальной компиляции формулируется следующим образом: пусть была произведена полная сборка, затем исходные файлы (и, возможно, make-файл) были изменены; тогда при выполнении некоторых условий результат произведённой после этого основанной на make инкрементальной компиляции эквивалентен результату повторной полной сборки, если бы такая была произведена. С целью формулировки этих условий для make строится семантическая модель. Ключевым результатом работы является сформулированный набор критериев, которым должны удовлетворять правила make-файла, чтобы обеспечить вышеупомянутую безопасность. Помимо этого, установлены условия, при которых make-файл может быть модифицирован с сохранением свойства безопасности.

Хотя сформулированные критерии и являются довольно интуитивными, полученный результат представляет собой формальное обоснование существующей практики использования make. Кроме того, показано, что на основанную на make инкрементальную компиляцию можно полагаться и в некоторых неочевидных ситуациях, например, при определённых модификациях make-файла.

В работах \cite{amake2012}, \cite{amake2013}/