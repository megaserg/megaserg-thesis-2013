\section{Обзор литературы}

В работе \cite{safeness_niels} предпринята попытка формализовать подход к сборке, реализуемый утилитой \texttt{make}. Анализируется проблема безопасности использования \texttt{make} для инкрементальной компиляции как альтернативы полной сборке (brute-force build). Безопасность (safeness) инкрементальной компиляции формулируется следующим образом: пусть была произведена полная сборка, затем исходные файлы (и, возможно, \texttt{make}-файл) были изменены; тогда при выполнении некоторых условий результат произведённой после этого основанной на \texttt{make} инкрементальной компиляции эквивалентен результату повторной полной сборки, если бы такая была произведена. С целью формулировки этих условий для \texttt{make} строится семантическая модель. Ключевым результатом работы является сформулированный набор критериев, которым должны удовлетворять правила \texttt{make}-файла, чтобы обеспечить вышеупомянутую безопасность. Помимо этого, установлены условия, при которых \texttt{make}-файл может быть модифицирован с сохранением свойства безопасности.

Хотя сформулированные критерии и являются довольно интуитивными, полученный результат представляет собой формальное обоснование существующей практики использования \texttt{make}. Кроме того, показано, что на основанную на \texttt{make} инкрементальную компиляцию можно полагаться и в некоторых неочевидных ситуациях, например, при определённых модификациях \texttt{make}-файла.\\

В работах \cite{amake2012}, \cite{amake2013} предлагается модификация утилиты \texttt{make} под названием Amake, реализующая подход, при котором система сборки, основанная на \texttt{make}-файлах (Makefile-based build system), улучшается с помощью автоматического анализа зависимостей. Помимо стандартных зависимостей между файлами (а именно, зависимостей целевых файлов от исходных и других целевых), отслеживаются следующие зависимости: команды оболочки ОС (shell), используемые для генерации целевых файлов; программы, исполнение которых предписывается правилами; библиотеки общего пользования (shared libraries), используемые программами, упомянутыми в правилах; переменные окружения (environment variables). Вместо временных меток для определения изменившихся зависимостей используются подсчитываемые для файлов хэш-суммы. Также хэш-суммы подсчитываются для содержимого правил, а также для исполняемых файлов упоминаемых в них программ, с помощью чего отслеживаются их изменения. Для автоматического отслеживания изменившихся файловых зависимостей применяется перехват вызовов методов (таких как \texttt{open()}) стандартной библиотеки C (Standard C Library); для этого используется переменная окружения \texttt{LD\_PRELOAD}, что сужает область применения Amake до Unix-подобных ОС. В архитектуре выделяется кэш, в котором хранятся целевые файлы, порождённые предыдущими сеансами компиляции. Содержимое кэша предлагается хранить на сетевом NFS-диске, а индекс~--- в реляционной базе данных. Это допускает одновременное использование системы несколькими разработчиками. Отмечается, что использование хэш-сумм лишь незначительно медленнее сравнения временных меток.\\

В рассмотренных работах прослеживаются как попытки формального описания предметной области и построения на этой базе критериев безопасности переиспользования порождений (т.е. намерения определить теоретические границы применимости подхода), так и технические решения, направленные на автоматизацию и ускорение процесса сборки, а также учитывающие переносимость и одновременное участие в проекте нескольких разработчиков. В силу того, что в качестве основной системы сборки рассматривается основанная на \texttt{make}-файлах, остаётся пространство для обобщения и построения более абстрактных формальных систем, а также базирующихся на них технических решений.