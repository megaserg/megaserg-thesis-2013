\documentclass[a4paper,12pt]{report} %%%{article}

\usepackage{cmap} % searchable PDFs
\usepackage[T2A]{fontenc} % scalable fonts
\usepackage[utf8]{inputenc} % input in UTF8
\usepackage[english,russian]{babel} % dashes on linebreaks
\usepackage{indentfirst} % indents in paragraphs
\usepackage{amstext,amssymb,amsfonts,amsmath,mathtext,enumerate,float}
\usepackage[left=25mm,right=2cm,top=2cm,bottom=2cm,bindingoffset=0cm]{geometry}
\usepackage[unicode]{hyperref}
\usepackage{graphicx}
\usepackage{ulem} % strikethrough
\usepackage{verbatim} % multiline comments
\usepackage{hhline}

\usepackage{listings}
\usepackage{color}
 
\lstset{extendedchars=\true}
  
\definecolor{dkgreen}{rgb}{0,0.6,0}
\definecolor{gray}{rgb}{0.5,0.5,0.5}
\definecolor{mauve}{rgb}{0.58,0,0.82}
 
\lstset{ %
  columns=flexible,
%  language=C,                     % the language of the code
  basicstyle=\footnotesize\ttfamily,       % the size of the fonts that are used for the code
  numbers=left,                   % where to put the line-numbers
  numberstyle=\tiny\color{gray},  % the style that is used for the line-numbers
  stepnumber=1,                   % the step between two line-numbers. If it's 1, each line will be numbered
  numbersep=5pt,                  % how far the line-numbers are from the code
  backgroundcolor=\color{white},  % choose the background color. You must add \usepackage{color}
  showspaces=false,               % show spaces adding particular underscores
  showstringspaces=false,         % underline spaces within strings
  showtabs=false,                 % show tabs within strings adding particular underscores
%  frame=single,                   % adds a frame around the code
  rulecolor=\color{black},        % if not set, the frame-color may be changed on line-breaks within not-black text (e.g. comments (green here))
  tabsize=4,                      % sets default tabsize
  captionpos=b,                   % sets the caption-position to bottom
  breaklines=true,                % sets automatic line breaking
  breakatwhitespace=false,        % sets if automatic breaks should only happen at whitespace
  title=\lstname,                 % show the filename of files included with \lstinputlisting; also try caption instead of title
  keywordstyle=\color{blue},      % keyword style
  commentstyle=\color{dkgreen},   % comment style
  stringstyle=\color{mauve},      % string literal style
%  escapeinside={\%*}{*)},         % if you want to add LaTeX within your code
  mathescape=true,
  morekeywords={*,...},           % if you want to add more keywords to the set
  deletekeywords={...},           % if you want to delete keywords from the given language
  aboveskip=1em,
  belowskip=-1em
}


\usepackage{mathtools}
\newcommand{\defeq}{\vcentcolon=}
\newcommand{\eqdef}{=\vcentcolon}

\renewcommand{\contentsname}{Содержание} 
\setcounter{secnumdepth}{0}
\setcounter{tocdepth}{3}

% \linespread{1.3}
\sloppy % align=justify


\begin{document}

%
% Титульный лист на русском языке
%
\thispagestyle{empty}
\begin{center}
САНКТ-ПЕТЕРБУРГСКИЙ ГОСУДАРСТВЕННЫЙ УНИВЕРСИТЕТ\\
Математико-механический факультет\\
\end{center}

\begin{center}
Кафедра системного программирования\\
\end{center}
\vspace{2cm}
\begin{center}
    \LARGE{Схема сборки проектов с агрессивным переиспользованием порождений} \\
\end{center}
\vspace{1cm}
\begin{center}
    \normalsize{Дипломная работа студента 545 группы} \\
    \large{Серебрякова Сергея Николаевича}
\end{center}
\vspace{3cm}
\noindent
\begin{center}
    \small
    \begin{tabular}{lcl}
        Научный руководитель & \dotuline{\phantom{место для подписи}} & к.\,ф.-м.\,н.\\
        & /подпись/ & Булычев~Д.~Ю.\\\\
        Рецензент & \dotuline{\phantom{место для подписи}} & к.\,ф.-м.\,н.\\
        & /подпись/& Чашников~Н.~В.\\\\
        ``Допустить к защите'' & \dotuline{\phantom{место для подписи}} & д.\,ф.-м.\,н., профессор\\
        заведующий кафедрой, & /подпись/& Терехов~А.~Н.\\
    \end{tabular}
\end{center}
\vspace{\fill}
\begin{center}
    \small
    Санкт-Петербург\\2013
\end{center}

\newpage
%
% Титульный лист на английском языке
%
\thispagestyle{empty}
\begin{center}
SAINT PETERSBURG STATE UNIVERSITY\\
Mathematics \& Mechanics Faculty\\
\end{center}

\begin{center}
Software Engineering Chair\\
\end{center}
\vspace{2cm}
\begin{center}
    \LARGE{Project build technique with aggressive generation reuse} \\
\end{center}
\vspace{1cm}
\begin{center}
    \normalsize{Graduate paper} \\
	\normalsize{by} \\
    \large{Sergey Serebryakov}
\end{center}
\vspace{3cm}
\noindent
\begin{center}
    \small
    \begin{tabular}{lcl}
        Supervisor & \dotuline{\phantom{место для подписи}} & Ph.D., Associate Professor\\
        & /signature/ & Dmitry Boulytchev\\\\
        Reviewer & \dotuline{\phantom{место для подписи}} & Ph.D.\\
        & /signature/& Nikolay Chashnikov\\\\
        ``Approved by'' & \dotuline{\phantom{место для подписи}} & Ph.D., Professor\\
        Head of the Chair, & /signature/& Andrey Terekhov\\
    \end{tabular}
\end{center}
\vspace{\fill}
\begin{center}
    \small
    Saint Petersburg\\2013
\end{center}

\newpage
\tableofcontents
\newpage
%\documentclass[a4paper,12pt]{report} %%%{article}

\usepackage{cmap} % searchable PDFs
\usepackage[T2A]{fontenc} % scalable fonts
\usepackage[utf8]{inputenc} % input in UTF8
\usepackage[english,russian]{babel} % dashes on linebreaks
\usepackage{indentfirst} % indents in paragraphs
\usepackage{amstext,amssymb,amsfonts,amsmath,mathtext,enumerate,float}
\usepackage[left=25mm,right=2cm,top=2cm,bottom=2cm,bindingoffset=0cm]{geometry}
\usepackage[unicode]{hyperref}
\usepackage{graphicx}
\usepackage{ulem} % strikethrough
\usepackage{verbatim} % multiline comments
\usepackage{hhline}

\usepackage{listings}
\usepackage{color}
 
\lstset{extendedchars=\true}
  
\definecolor{dkgreen}{rgb}{0,0.6,0}
\definecolor{gray}{rgb}{0.5,0.5,0.5}
\definecolor{mauve}{rgb}{0.58,0,0.82}
 
\lstset{ %
  columns=flexible,
%  language=C,                     % the language of the code
  basicstyle=\footnotesize\ttfamily,       % the size of the fonts that are used for the code
  numbers=left,                   % where to put the line-numbers
  numberstyle=\tiny\color{gray},  % the style that is used for the line-numbers
  stepnumber=1,                   % the step between two line-numbers. If it's 1, each line will be numbered
  numbersep=5pt,                  % how far the line-numbers are from the code
  backgroundcolor=\color{white},  % choose the background color. You must add \usepackage{color}
  showspaces=false,               % show spaces adding particular underscores
  showstringspaces=false,         % underline spaces within strings
  showtabs=false,                 % show tabs within strings adding particular underscores
%  frame=single,                   % adds a frame around the code
  rulecolor=\color{black},        % if not set, the frame-color may be changed on line-breaks within not-black text (e.g. comments (green here))
  tabsize=4,                      % sets default tabsize
  captionpos=b,                   % sets the caption-position to bottom
  breaklines=true,                % sets automatic line breaking
  breakatwhitespace=false,        % sets if automatic breaks should only happen at whitespace
  title=\lstname,                 % show the filename of files included with \lstinputlisting; also try caption instead of title
  keywordstyle=\color{blue},      % keyword style
  commentstyle=\color{dkgreen},   % comment style
  stringstyle=\color{mauve},      % string literal style
%  escapeinside={\%*}{*)},         % if you want to add LaTeX within your code
  mathescape=true,
  morekeywords={*,...},           % if you want to add more keywords to the set
  deletekeywords={...},           % if you want to delete keywords from the given language
  aboveskip=1em,
  belowskip=-1em
}


\usepackage{mathtools}
\newcommand{\defeq}{\vcentcolon=}
\newcommand{\eqdef}{=\vcentcolon}

\renewcommand{\contentsname}{Содержание} 
\setcounter{secnumdepth}{0}
\setcounter{tocdepth}{3}

% \linespread{1.3}
\sloppy % align=justify


%\begin{document}

\section{Введение}
Сборка проекта в интегрированных средах разработки (Integrated Development Environment, IDE)~--- это процесс генерации низкоуровневых артефактов из исходных файлов высокого уровня (программного кода, ресурсов и т.п.) по правилам, описанным в проекте.

Кэш компиляции~--- это некоторая структура данных, которая отражает мгновенное состояние проекта с точки зрения его реализации на низком уровне (например, для языка Java это совокупность класс-файлов плюс информация о зависимостях, соответствиях класс-файлов исходным файлам и т.д.). Такое хранилище может быть использовано (и используется в современных средах разработки) в процессе инкрементальной компиляции.

Инкрементальная компиляция представляет собой следующую оптимизацию процесса сборки проекта: вместо полной компиляции всего проекта с нуля перекомпилируются только изменённые со времени последней компиляции файлы, а также зависимые от них, на которые повлияли изменения (а также зависимые от них, и так далее~--- можно представить это как обход графа зависимостей в ширину). Например, если в одном из исходных Java-файлов разработчик изменил сигнатуру некоторого метода, то нужно перекомпилировать не только этот файл, но и все файлы, где присутствует вызов этого метода (т.е. зависимые). В целях экономии вычислительных ресурсов информация о зависимостях подобного рода сохраняется в кэше компиляции, чтобы при следующем запуске сборки проекта можно было эффективно вычислять набор файлов, которые подлежат перекомпиляции. В этот же кэш попадают и класс-файлы~--- в случае, когда ни исходный файл, ни те файлы, от которых он зависит, не изменились со времени последней компиляции, ясно, что можно переиспользовать результат его последней компиляции~--- взять сохранённый класс-файл, вместо того чтобы тратить вычислительные ресурсы на компиляцию.

Кэш компиляции подвергается обновлению при каждой компиляции, чтобы находиться в актуальном состоянии, пригодном к использованию при следующем сеансе компиляции. Если над проектом работает несколько человек, у каждого из которых есть локальная рабочая копия репозитория, то у каждого ``нарастает'' свой локальный кэш компиляции, при этом б\'{о}льшая его часть у всех одинакова. В процессе такой коллективной разработки случаются ситуации, когда состояние проекта в локальной копии разработчика существенно отличается от состояния, для которого был посчитан его локальный кэш компиляции. Тогда инкрементальная компиляция теряет свои преимущества и мало чем отличается от процедуры полной сборки проекта с нуля. Примерами таких случаев могут служить ``cold start''~--- сборка проекта в ситуации отсутствия кэша вообще~--- или переключение между ветками (``branches'')~--- например, основной и релизной ветками. Однако можно воспользоваться тем, что разработчиков всё-таки несколько и предложить способ переиспользования кэшей одних разработчиков другими, чтобы в таких ситуациях не приходилось каждый раз строить их с нуля.

В данной работе предлагается такой способ переиспользования кэшей. Рассматривается главным образом процесс компиляции исходных файлов на языке Java, однако полученные результаты достаточно абстрактны для того, чтобы данный подход можно было применить к другим языкам программирования. Заметим, что описываемый подход применим не только к компиляции, но также и к другим вычислительно сложным процессам, которые получают на вход некоторый набор исходных файлов и выдают в качестве результата некоторый набор порождений (примером такого процесса может служить построение индексов).

%\end{document}
\newpage
\section{Постановка задачи}

В рамках дипломной работы были поставлены следующие задачи:

\begin{itemize}

	\item Построить формальную модель предметной области (сборка проекта, исходные файлы, сгенерированные файлы-порождения и т.п.)

	\item Построить аксиоматику, описывающую свойства модели, позволяющую доказывать нетривиальные утверждения и вместе с тем отражающую ограничения реального мира

	\item Описать задачу инкрементальной компиляции в терминах построенной формальной модели, сформулировать и доказать соответствующую теорему

	\item Сформулировать и доказать теорему о переиспользовании порождений

\end{itemize}
\newpage
\section{Обзор}

\subsection{Инкрементальная компиляция в IntelliJ IDEA}
В текущей реализации инкрементальной компиляции в IntelliJ IDEA используется локальное кэширование, изменение файлов отслеживается по timestamp~--- для каждого файла запоминается время его самого позднего изменения на момент последней компиляции, и файл считается изменившимся, если фактическое время его самого позднего изменения не совпадает с запомненным. В кэше хранятся абсолютные пути к файлам, таким образом, при перемещении проекта или его части в другую директорию кэш теряется. В ситуации, когда разработчиков несколько, временные метки в силу их относительности использовать уже нельзя. Предлагается вместо них применять хэширование исходных файлов, а именно хэширующую структуру, способную считать контрольные суммы не только на уровне отдельных файлов, но и на уровне директорий и модулей проекта.

\subsection{Инкрементальная компиляция в Eclipse}

\subsection{Инструмент ccache}
\newpage
\section*{Заключение}
В качестве возможного продолжения работы планируется реализовать прототип, демонстрирующий работоспособность подхода на базе доказанных теорем. В качестве основы для реализации прототипа предполагается использовать среду разработки IntelliJ IDEA\footnote{\url{http://www.jetbrains.com/idea/}}, разрабатываемой компанией JetBrains. Чтобы проверить работоспособность и эффективность предложенного подхода, предлагается провести на реализованном прототипе эксперименты, в ходе которых дать ответы на следующие вопросы: В каких случаях эффективно кэширование? В каких выгоднее компилировать с нуля? Что быстрее~--- передача кэшированных класс-файлов по сети или локальная перекомпиляция? В ходе экспериментов требуется рассмотреть различные операционные системы, различные файловые системы, различные дисковые носители (HDD vs. SSD), а также варьировать размеры проектов и размеры данных, пересылаемых по сети.

Полученные результаты являются достаточно общими для того, чтобы подход можно было применить и к другим языкам программирования. Заметим, что подход применим не только к компиляции, но также и к другим вычислительно сложным процессам, которые получают на вход некоторый набор исходных файлов и выдают в качестве результата некоторый набор порождений. Примером может служить задача переиспользования индексов, используемых средами разработки для реализации интеллектуальных функций вроде ``Find Usages'', ``Find Implementations'', рефакторинга и т.п. Построение таких индексов с нуля для крупных проектов занимает достаточно много времени, поскольку требует обхода и чтения всех исходных файлов проекта. Используя идеи, сходные с упомянутыми для задачи переиспользования порождений, можно добиться существенного сокращения времени построения этих индексов.

\begin{comment}
\newpage

\chapter{Формальная модель}

Пусть $\Sigma$~--- множество входов, $\Omega$~--- множество всех возможных порождений. Выделим специальное множество $ERROR \subset \Omega$\footnote{Здесь и далее $\subset$ означает ``является собственным подмножеством'', а $\subseteq$~--- ``равно или является собственным подмножеством''.}.\\
Для $\sigma \subset \Sigma$, $s \in \Sigma$ определим $gen_\sigma(s) \subset \Omega$~--- порождения $s$ в контексте $\sigma$. Таким образом, тип функции $gen: 2^\Sigma \times \Sigma \to 2^\Omega$.\\
Контекст $\sigma$ назовём \textit{необходимым} для входа $s$, если для любого множества $\varphi: \varphi \subset \sigma$ $gen_\varphi(s) \subseteq ERROR$ или $gen_\varphi(s) = \varnothing$.\\
Контекст $\sigma$ назовём \textit{достаточным} для входа $s$, если для любого множества $\Phi: \sigma \subseteq \Phi$ $gen_\Phi(s) = gen_\sigma(s)$.\\
Доопределим $gen$ для набора $S \subset \Sigma$: $gen_\sigma(S) \defeq \bigcup\limits_{s \in S} gen_\sigma(s)$.\\
В частности, $gen_\sigma(\sigma) = \bigcup\limits_{s \in \sigma} gen_\sigma(s)$~--- порождение набора входов в контексте этого же набора.\\
Пусть для каждого набора $\sigma \subset \Sigma$, входа $s \in \sigma$ задана функция $d(\sigma, s) \subseteq \sigma$~--- некоторый необходимый и достаточный контекст для входа $s$. TODO: что если таких несколько?\\
Тогда $gen^d(\sigma) \defeq \bigcup\limits_{s \in \sigma} gen_{d(\sigma, s)}(s)$.

\subsection{Задача инкрементальной компиляции}

\newcommand{\Deltasigma}{\Delta_\alpha^\rho\sigma}
Рассмотрим $\Deltasigma = (\sigma \setminus \rho) \cup \alpha$, где $\rho, \alpha \subset \Sigma$, при этом $\rho \subseteq \sigma$, $\sigma \cap \alpha = \varnothing$. Тогда
$$gen^d(\Deltasigma) \overset{def}{=} \bigcup\limits_{s \in \Deltasigma} gen_{d(\Deltasigma, s)}(s) = \bigcup\limits_{s \in (\sigma \setminus \rho) \cup \alpha} gen_{d((\sigma \setminus \rho) \cup \alpha, s)}(s).$$
Требуется выразить эту функцию через $gen^d(\sigma)$.

\subsection{Задача переиспользования порождений других входов}

Пусть есть $\sigma_1, \dots, \sigma_n$ ($\sigma_i \subset \Sigma$)~--- наборы входов и известны $\omega_1, \dots, \omega_n$, где $\omega_i = gen^d(\sigma_i)$.\\
Также есть набор $\sigma^*$, при этом $\sigma^* \subset \bigcup\limits_{i \in [1:n]}\sigma_i$, т.е. $\forall s \in \sigma^*$ $\exists i_s \in [1:n]: s \in \sigma_{i_s}$.\\
Требуется получить $gen^d(\sigma^*)$, используя известные порождения $\omega_1, \dots, \omega_n$.

\section{Случай компиляции}

Пусть у нас есть основной репозиторий, в котором лежат исходные файлы. Содержимое основного репозитория меняется от ревизии к ревизии. Также есть несколько локальных копий репозитория, каждая из которых была синхронизирована с репозиторием в какой-то момент времени и, возможно, имеет какие-то локальные изменения.\\

Параметризуем наши сущности моментом времени $t$, но будем опускать его при записи. Пусть $n = n^t$~--- количество файлов в основном репозитории в момент  $t$, $s = s^t$~--- количество ревизий/версий в основном репозитории в момент $t$ (можно считать, что они упорядочены по времени создания, т.е. номер ревизии возрастает со временем), $r = r^t$~--- количество локальных копий основного репозитория в момент $t$.\\

Будем рассматривать два случая: 1) когда для каждого файла история его изменений хранится отдельно, но нумерация ревизий всё равно сквозная (будем называть его ``случай SVN'') и 2) когда каждая ревизия~--- снимок всего репозитория (будем называть его ``случай Git''). В первом случае для каждого файла в любой момент известен номер последней ревизии, в которой он изменялся; во втором случае каждый файл ``обновляется'' в каждой ревизии, но это изменение ненулевое только тогда, когда файл изменялся фактически.\\

Обозначим $SRC \subset \{0,1\}^*$~--- множество всех возможных исходных файлов, $CF \subset \{0,1\}^*$~--- множество всех возможных класс-файлов, $ERR$~--- множество всех возможных ошибок компиляции.
Пусть $FN = FN^t = \{FN_1, \dots, FN_n\}$~--- пути к файлам в репозитории в момент $t$ (или можно дополнительно ввести набор директорий, понятия вложенности директорий и принадлежности файла директории~--- тогда $FN_i$ будет не путём к файлу, а его именем).\\

Введём обозначения для информации, хранящейся для файлов в основном репозитории.\\
В случае SVN:\\
$\forall i \in [1:n]$\\
\indent a) $cont(i) = cont^t(i) \in SRC$~--- содержимое файла $i$ в момент $t$;\\
\indent b) $ts(i) = ts^t(i)$~--- timestamp последнего изменения файла $i$ в момент $t$;\\
\indent c) $lv(i) = lv^t(i) \in [1:s]$~--- последняя на момент $t$ ревизия, когда изменялся файл $i$.\\
В случае Git:\\
$\forall v \in [1:s], \forall i \in [1:n]$\\
\indent a) $cont_v(i) \in SRC$~--- содержимое файла $i$ в ревизии $v$;\\
\indent b) $ts_v(i)$~--- timestamp последнего изменения файла $i$ в ревизии $v$.\\

Похожим образом обозначим информацию, хранящуюся для файлов в локальных копиях.\\
Пусть $R = R^t = \{R_1, \dots, R_r\}$~--- локальные копии репозитория.\\
$\forall l \in [1:r], \forall i \in [1:n]$\\
\indent a) $loccont_l(i) = loccont_l^t(i) \in SRC$~--- содержимое файла $i$ в локальной копии $l$ в момент $t$;\\
\indent b) $locts_l(i) = locts_l^t(i)$~--- timestamp последнего изменения файла $i$ в локальной копии $l$ в момент $t$;\\
\indent c) $loclv_l(i) = loclv_l^t(i) \in {[1:s]}$~--- последняя на момент $t$ ревизия основного репозитория, с которой был синхронизирован файл $i$ в локальной копии $l$;\\
(В случае Git $\forall i \in [1:n]$ $loclv_l(i)$ одинаковы~--- это последняя на момент $t$ ревизия основного репозитория, с которой была синхронизирована локальная копия $l$.)\\
\indent d) $lochash_l(i) = lochash_l^t(i) = hash(loccont_l^t(i)) $~--- хэш содержимого файла $i$ в локальной копии $l$ в момент $t$.

Таким образом, локальное изменение файла $i$ в копии $l$ (в момент $t$): $$\mathit{diff}(loccont_l(i), cont_{loclv_l(i)}(i)),$$ где $\mathit{diff}$~--- функция вычисления разности двух исходных файлов.\\
TODO: случай несуществующего файла.

\subsection{Зависимости}

Пусть $V = \{v_1, \dots, v_n\}$~--- набор файлов в состоянии некоторого репозитория (основного или локальной копии) в некоторый момент времени, где каждое $v_i$ является парой $(path_i, cont_i)$, которая означает, что в файле по пути $path_i$ в этом репозитории в этот момент времени записано содержимое $cont_i \in SRC$. Пусть мы умеем считать зависимости для файлов; тогда можно построить набор зависимостей $E(V) = \{e_1, \dots, e_m\}$, где каждое $e_j$ является тройкой $(\textit{type}_j, \textit{src}_j, \textit{dst}_j)$, которая означает, что файл $\textit{src}_j$ зависит от файла $\textit{dst}_j$, при этом тип зависимости $\textit{type}_j$.\\

Тогда рассмотрим $\Sigma = V, \Omega = CF \cup ERR$. $\forall \sigma \subseteq V$, $\forall s \in V$ можем задать функцию $gen_\sigma(s) = \textit{compile}(\sigma, s) \subset CF \cup ERR$~--- результат компиляции файла $s$ при условии, что рядом в SOURCEPATH лежат файлы $\sigma$. Можем определить функцию $d(V,s) \subseteq V$~--- набор файлов из $V$, необходимый для безошибочной компиляции файла $s$ и при этом достаточный в том смысле, что от добавления в него других файлов результат компиляции не меняется. (TODO: верно ли, что при фиксированных $V$ и $s \in V$ $\forall \sigma \subseteq V$ либо $\textit{compile}(\sigma, s) \subset ERR$, либо $\textit{compile}(\sigma, s) = FILES_s \subset CF$, то есть для любого контекста компиляция выдаёт либо один и тот же результат, либо ошибку?) Эта функция может быть определена как транзитивное замыкание функции $dependencies$, возвращающей непосредственные зависимости и определяемой как $dependencies(V,s) \defeq \{v \mid \exists e_j \in E(V): src_j = s, dst_j = v\}$. (TODO: зависимости надо считать с учётом контекста?)

\subsection{Локальный кэш на timestamp}

Пусть $ct_l^t$~--- момент, когда в локальной копии $l$ была произведена компиляция, являющаяся последней на момент $t$. Например, пусть компиляция была произведена в некоторый момент $x$, тогда $ct_l^t$ будет равно $x$ для всех $t \geqslant x$, пока не будет произведена следующая компиляция.\\
$\forall l \in [1:r], \forall i \in [1:n]$\\
\indent a) $loclct_l^t(i) = locts_l^{ct_l^t}(i)$~--- timestamp последнего на момент $ct_l^t$ изменения файла $i$ в локальной копии $l$;\\
\indent b) $loccache_l^t(i) = compile(..., loccont_l^{ct_l^t}(i))$~--- результат последней компиляции файла $i$ в локальной копии $l$, то есть компиляции его содержимого на момент $ct_l^t$.\\

\subsection{Локальный кэш на хэшах}

Рассмотрим теперь локальный кэш на хэшах~--- некоторое хранилище, ассоциированное с локальной копией, в каждый момент времени содержащее результаты некоторых последних компиляций. Опять, пусть $ct_l^t$~--- момент, когда в локальной копии $l$ была произведена компиляция, являющаяся последней на момент $t$.
$\forall l \in [1:r], \forall i \in [1:n]$\\
\indent a) $loclh_l^t(i) = lochash_l^{ct_l^t}(i)$~--- хэш содержимого файла $i$ в локальной копии $l$ на момент последней компиляции, то есть $ct_l^t$;\\
\indent b) $loccache_l^t(i) = compile(..., loccont_l^{ct_l^t}(i))$~--- результат последней компиляции файла $i$ в локальной копии $l$, то есть компиляции его содержимого на момент $ct_l^t$.\\

Пусть есть некоторая локальная копия $l$. Тогда можно описать, что происходит с переменными, описывающими содержащуюся в ней информацию, при некоторых процессах.\\

\subsubsection[Функция \texttt{rebuild}]{Функция \texttt{rebuild}: полная компиляция}
\begin{lstlisting}
function $\mathbf{rebuild}(l)$:
	for $\forall i \in [1:n]$ do
		$loclct_l(i) \defeq locts_l(i)$
		$loclh_l(i) \defeq lochash_l(i)$
		$loccache_l(i) \defeq compile(..., loccont_l(i))$
		$result[i] \defeq loccache_l(i)$
	return $result$
\end{lstlisting}

\subsubsection[Функция \texttt{make\_ts}]{Функция \texttt{make\_ts}: инкрементальная компиляция с использованием локального кэша на timestamp}
\begin{lstlisting}
function $\mathbf{make\_ts}(l)$:
	$tocompile \defeq \varnothing$
	$result \defeq \varnothing$
	for $\forall i \in [1:n]$ do
		$result[i] \defeq loccache_l(i)$
	for $\forall i \in [1:n]$ do
		if $locts_l(i) \neq loclct_l(i)$ then
			$tocompile \defeq tocompile \cup \{i\} \cup dependents(i)$
	for $\forall i \in tocompile$ do
		$loclct_l(i) \defeq locts_l(i)$
		$loccache_l(i) \defeq compile(..., loccont_l(i))$
		$result[i] \defeq loccache_l(i)$
	return $result$
\end{lstlisting}

\subsubsection[Функция \texttt{make\_hash}]{Функция \texttt{make\_hash}: инкрементальная компиляция с использованием локального кэша на хэшах}
\begin{lstlisting}
function $\mathbf{make\_hash}(l)$:
	$tocompile \defeq \varnothing$
	$result \defeq \varnothing$
	for $\forall i \in [1:n]$ do
		$result[i] \defeq loccache_l(i)$
	for $\forall i \in [1:n]$ do
		if $lochash_l(i) \neq loclh_l(i)$ then
			$tocompile \defeq tocompile \cup \{i\} \cup dependents(i)$
	for $\forall i \in tocompile$ do
		$loclh_l(i) \defeq lochash_l(i)$
		$loccache_l(i) \defeq compile(..., loccont_l(i))$
		$result[i] \defeq loccache_l(i)$
	return $result$
\end{lstlisting}

\subsection{Глобальный кэш на хэшах}

Рассмотрим теперь глобальный, или удалённый, кэш на хэшах~--- некоторое хранилище, содержащее результаты компиляций для каждой ревизии в центральном репозитории.
$\forall v \in [1:s], \forall i \in [1:n]$\\
\indent a) $cont_v(i) \in SRC$~--- содержимое файла $i$ в ревизии $v$;\\
\indent b) $hash_v(i)$~--- хэш содержимого файла $i$ в ревизии $v$;\\
\indent c) $cache_v(i) = compile(..., cont_v(i))$~--- результат компиляции файла $i$ в ревизии $v$.\\

\subsubsection[Функция \texttt{make\_glob\_hash}]{Функция \texttt{make\_glob\_hash}: инкрементальная компиляция с использованием локального и глобального кэшей на хэшах}
Пусть есть локальная копия репозитория, локальный кэш на хэшах и глобальный (удалённый) кэш на хэшах. Рассмотрим ситуацию отсутствия зависимостей, тогда для каждого файла можно независимо считать хэш и порождения.
Зафиксируем некоторый файл. Пусть $hash$~--- хэш текущей локальной копии файла, $h1$~--- хэш файла на момент последнего вызова функции порождения, $c1$~--- результат последнего вызова функции порождения, $h2$~--- хэш в удалённом кэше, $c2$~--- результат компиляции, хранящийся в удалённом кэше.

\begin{lstlisting}
// $\text{Вызов функции порождения на текущей локальной копии (и обновление локального кэша).}$
function $\mathbf{regenerate}$:
	$h1 \defeq hash$
	$c1 \defeq gen(src)$
\end{lstlisting}
\begin{lstlisting}
// $\text{Обновление удалённого кэша.}$
function $\mathbf{upload}$:
	$h2 \defeq h1$
	$c2 \defeq c1$
\end{lstlisting}
\begin{lstlisting}
// $\text{Копирование удалённого кэша в локальный.}$
function $\mathbf{download}$:
	$h1 \defeq h2$
	$c1 \defeq c2$
\end{lstlisting}

Тогда можем описать функцию, которая использует кэши и не вызывает функцию порождения, если это не необходимо.

\begin{lstlisting}
// $\text{Версия 1}$
function $\mathbf{make\_glob\_hash}$:
	if $\exists h2$ then
		if $\exists h1$ then
			if $hash = h1$ then
				if $hash = h2$ then
					// nothing to do
				else
					upload
			else
				if $hash = h2$ then
					download (or regenerate)
				else
					regenerate
					upload
		else
			if $hash = h2$ then
				download (or regenerate)
			else
				regenerate
				upload
	else
		if $\exists h1$ then
			if $hash = h1$ then
				upload
			else
				regenerate
				upload
		else
			regenerate
			upload
\end{lstlisting}

Её можно переписать в более коротком виде:
\begin{lstlisting}
// $\text{Версия 2}$
function $\mathbf{make\_glob\_hash}$:
	if $\exists h2 \wedge hash = h2$ then
		if $\nexists h1 \vee hash \neq h1$ then
			download (or regenerate)
	else
		if $\nexists h1 \vee hash \neq h1$ then
			regenerate
		upload
\end{lstlisting}

\subsection{TODO}
Далее можно описать процедуры \texttt{commit} и \texttt{checkout}, процедуры добавления, удаления и изменения файлов.

\newpage

\subsection{Задача инкрементальной компиляции}

\newcommand{\Deltasigma}{\Delta_\alpha^\rho\sigma}
Рассмотрим $\Deltasigma = (\sigma \setminus \rho) \cup \alpha$, где $\rho, \alpha \subset \Sigma$, при этом $\rho \subseteq \sigma$, $\sigma \cap \alpha = \varnothing$. Тогда
$$gen^d(\Deltasigma) \overset{def}{=} \bigcup\limits_{s \in \Deltasigma} gen_{d(\Deltasigma, s)}(s) = \bigcup\limits_{s \in (\sigma \setminus \rho) \cup \alpha} gen_{d((\sigma \setminus \rho) \cup \alpha, s)}(s).$$
Требуется выразить эту функцию через $gen^d(\sigma)$.

\hrulefill

Введём:
$$P = \{s \in \sigma \mid s \in \rho \vee d(\sigma, s) \cap \rho \neq \varnothing \},$$
$$A = \{s \mid s \in \alpha \vee d(\Deltasigma, s) \cap \alpha \neq \varnothing \} \cup (P \setminus \rho),$$

Сделаем несколько допущений-аксиом:\\
\indent \textbf{a)} для произвольных $\sigma$, $x$ верно, что $(\exists s: x \in gen_\sigma(s))$ $\Rightarrow$ $\forall t \neq s$ $x \notin gen_\sigma(t)$ (в одном и том же контексте порождение может принадлежать результату генерации для не более чем одного входа).\\
\indent \textbf{b)} $\forall s \in \{s \in \sigma \setminus \rho \mid d(\sigma, s) \cap \rho = \varnothing \wedge d(\Deltasigma, s) \cap \alpha = \varnothing \}$ $gen_{d(\sigma, s)} (s) = gen_{d(\Deltasigma, s)} (s)$ (если ни сам вход, ни его зависимости не лежали в изменяющейся части, результат генерации не меняется).\\
\indent \textbf{c)} $d(\sigma, s)$ не определено, если для $s$ невозможно найти подмножество $\sigma$, являющееся необходимым и достаточным контекстом (из-за отсутствующих в контексте или дублирующих друг друга зависимостей); в таком случае $gen_{d(\sigma, s)}(s) = \varnothing$.\\
\indent \textbf{d)} если $d(\sigma, s)$ определено, то $\forall \tau$ таких, что $\sigma \cap \tau = \varnothing$, $d(\sigma \cup \tau, s)$ либо равно $d(\sigma, s)$, либо не определено.\\

\textbf{Следствие} из допущения (d): если $d(\sigma, s)$ определено, то $\forall \sigma' \subseteq \sigma$ $d(\sigma', s)$ либо равно $d(\sigma, s)$, либо не определено.

\textbf{Доказательство}: Рассмотрим $\sigma' \subseteq \sigma$. Пусть $\tau = \sigma \setminus \sigma'$, тогда $\sigma' \cup \tau = \sigma$. По допущению (d), если $d(\sigma', s)$ определено, то $d(\sigma, s)$ либо равно $d(\sigma', s)$, либо не определено. Это эквивалентно следующему: если $d(\sigma, s)$ и $d(\sigma', s)$ определены, то они равны. Это эквивалентно следующему: если $d(\sigma, s)$ определено, то $d(\sigma', s)$ либо равно $d(\sigma, s)$, либо не определено. Конец.\\

Таким образом, допущение (d) означает, что контекст, передаваемый функции $d$ для вычисления необходимого и достаточного контекста, можно сужать и расширять без изменения результата до тех пор, пока функция $d$ определена. Или: если при фиксированном $s$ функция $d$ определена для некоторого контекста и его подмножества (надмножества), то значения функции на этих контекстах равны. Мы также считаем, что если ``хороший'' контекст (на котором $d$ определена) при расширении становится ``плохим'', то этот ``плохой'' нельзя расширить так, чтобы он стал ``хорошим''.\\

\indent \textbf{e)} если $d(\sigma \cup \tau, s)$ определено и $d(\sigma \cup \tau, s) \cap \tau = \varnothing$, то $d(\sigma, s)$ определено (и по допущению (d) равно $d(\sigma \cup \tau, s)$).\\

\textbf{Замечание}: допущение (e) эквивалентно следующему: если $d(\gamma, s)$ определено, то $d(d(\gamma, s), s)$ определено и равно $d(\gamma, s)$.

\textbf{Доказательство}: В одну сторону: зафиксируем $s$. Пусть для произвольных $\sigma$, $\tau$ верно, что из того факта, что $d(\sigma \cup \tau, s)$ определено и не пересекается с $\tau$, следует определённость $d(\sigma, s)$. Тогда рассмотрим произвольное $\gamma$ и положим $\sigma = d(\gamma, s)$, $\tau = \gamma \setminus d(\gamma, s)$. Тогда по посылке из определённости $d(\gamma, s)$ следует определённость $d(\sigma, s) = d(d(\gamma, s), s)$, притом они равны.\\
Обратно: зафиксируем $s$. Пусть для произвольного $\gamma$ верно, что если $d(\gamma, s)$ определено, то $d(d(\gamma, s), s)$ определено и равно $d(\gamma, s)$. Тогда рассмотрим произвольные $\sigma$, $\tau$ и положим $\gamma = \sigma \cup \tau$. Предположим, что определено $d(\sigma \cup \tau, s) = d(\gamma, s)$ и $d(\sigma \cup \tau, s) \cap \tau = \varnothing$. Тогда по посылке $d(\gamma, s) = d(d(\gamma, s), s) = d(d(\sigma \cup \tau, s), s)$. Получается, что $d(\cdot, s)$ определена для множества $d(\sigma \cup \tau, s)$, которое не пересекается с $\tau$ и потому является подмножеством $\sigma$; и для множества $\sigma \cup \tau$, которое является надмножеством $\sigma$. Тогда и $d(\sigma, s)$ определено.
Конец.\\

Если рассмотреть функцию от $\sigma$ $d_s(\sigma) = d(\sigma, s)$, то по замечанию получится $d_s(d_s(\sigma)) = d_s(\sigma)$, то есть функция зависимости для фиксированного входа $s$ идемпотентна, а необходимый и достаточный контекст является её наименьшей по включению неподвижной точкой.\\

\textbf{Предположение}: пусть $s \in \sigma \setminus P$. Тогда если $d(\sigma, s)$ определено и $d(\Deltasigma, s)$ определено, то $d(\Deltasigma, s) \cap \alpha = \varnothing$.

\textbf{Доказательство}: Рассмотрим произвольное $s \in \sigma \setminus P$. Для него определено $d(\sigma, s)$. Так как $s \notin P$, то $d(\sigma, s) \cap \rho = \varnothing$ и по допущению (e) $d(\sigma \setminus \rho, s)$ определено (и по допущению (d) равно $d(\sigma, s)$). Далее, рассмотрим $d(\Deltasigma, s) = d((\sigma\setminus\rho)\cup\alpha, s)$. Поскольку и $d(\sigma \setminus \rho, s)$, и $d((\sigma\setminus\rho)\cup\alpha, s)$ определены, то по допущению (d) они равны. Таким образом, $d(\Deltasigma, s) = d(\sigma, s) \subseteq \sigma \cap \alpha = \varnothing$. Конец.

\hrulefill

\textbf{Гипотеза}: пусть
$$P = \{s \in \sigma \mid s \in \rho \vee d(\sigma, s) \cap \rho \neq \varnothing \},$$
$$A = \{s \mid s \in \alpha \vee d(\Deltasigma, s) \cap \alpha \neq \varnothing \} \cup (P \setminus \rho),$$
тогда
$$gen^d(\Deltasigma) = gen^d(\sigma) \setminus \bigcup\limits_{s \in P} gen_{d(\sigma, s)}(s) \cup \bigcup\limits_{s \in A} gen_{d(\Deltasigma, s)}(s)$$
(считается, что $d(\Deltasigma, s)$ определено $\forall s \in \Deltasigma$)

\textbf{Доказательство:}\\
Пусть некоторый $x \in gen^d(\Deltasigma)$. Это означает, что $\exists s_x \in (\sigma \setminus \rho) \cup \alpha$: $x \in gen_{d(\Deltasigma, s_x)} (s_x)$.\\
Нужно доказать, что $x$ принадлежит правой части, то есть:
$$
\left[
\begin{aligned}
	&\left\{
	\begin{aligned}
		\exists a \in \sigma&: x \in gen_{d(\sigma, a)} (a) &(1)\\
		\forall b \in P&: x \notin gen_{d(\sigma, b)} (b) &(2)\\
		% TODO: make normal enumeration here
	\end{aligned}
	\right.\\
	&\exists c \in A: x \in gen_{d(\Deltasigma, c)} (c)\\
\end{aligned}
\right.
$$

На самом деле систему можно переписать:

$$
\left[
\begin{aligned}
	&\exists a \in \sigma \setminus P: &x \in gen_{d(\sigma, a)} (a)\\
	&\exists c \in A: &x \in gen_{d(\Deltasigma, c)} (c)\\
\end{aligned}
\right.
$$

В силу допущения (a) верность первого утверждения обеспечит верность для (1) и (2).\\
Рассмотрим два случая:\\
\indent a) $s_x \in \alpha$. Тогда предъявим $c = s_x$; $c \in \alpha \subseteq A$ и, действительно, $x \in gen_{d(\Deltasigma, c)} (c)$.\\
\indent b) $s_x \in \sigma \setminus \rho$. Тогда если $d(\sigma, s_x) \cap \rho \neq \varnothing$ или $d(\Deltasigma, s_x) \cap \alpha \neq \varnothing$, то опять предъявим $c = s_x \in A$. В противном случае воспользуемся допущением (b) и предъявим $a = s_x$.\\

Обратно: пусть некоторый $x$ принадлежит правой части. Нужно доказать, что $x$ принадлежит левой части, то есть $\exists s_x \in \Deltasigma$: $x \in gen_{d(\Deltasigma, s_x)} (s_x)$.\\
Рассмотрим два случая:\\
\indent a) $\exists c \in A: x \in gen_{d(\Deltasigma, c)} (c)$. Ясно, что $A \subseteq \Deltasigma$. Тогда предъявим $s_x = c \in \Deltasigma$. Действительно, $x \in gen_{d(\Deltasigma, s_x)} (s_x)$.\\
\indent b) $\forall c \in A: x \notin gen_{d(\Deltasigma, c)} (c)$, но $\exists a \in \sigma \setminus P: x \in gen_{d(\sigma, a)} (a)$. Ясно, что $\sigma \setminus P \subseteq \Deltasigma$. Тогда предъявим $s_x = a \in \Deltasigma$. По предположению: поскольку $d(\sigma, s_x)$ определено (что следует из непустоты $gen_{d(\sigma, a)}$) и $d(\Deltasigma, s_x)$ определено (что следует из условия гипотезы), то $d(\Deltasigma, s_x) \cap \alpha = \varnothing$. Тогда можно воспользоваться допущением (b) и получить $gen_{d(\sigma, a)} (a) = gen_{d(\Deltasigma, s_x)} (s_x)$ и $x \in gen_{d(\Deltasigma, s_x)} (s_x)$.\\

Конец. TODO: проверить на проблемы с неопределённостью

\newpage

\subsection{Задача переиспользования порождений других входов}

Пусть есть $\sigma_1, \dots, \sigma_n$ ($\sigma_i \subset \Sigma$)~--- наборы входов и известны $\omega_1, \dots, \omega_n$, где $\omega_i = gen^d(\sigma_i)$.\\
Также есть набор $\sigma^*$, при этом $\sigma^* \subset \bigcup\limits_{i \in [1:n]}\sigma_i$, т.е. $\forall s \in \sigma^*$ $\exists i_s \in [1:n]: s \in \sigma_{i_s}$.\\
Требуется получить $gen^d(\sigma^*)$, используя известные порождения $\omega_1, \dots, \omega_n$.\\

Пусть

$$\sigma_i' = \sigma_i \cap \sigma^*,$$
$$\rho_i = \sigma_i \setminus \sigma^* = \sigma_i \setminus \sigma_i',$$
$$\alpha_i = \sigma^* \setminus \sigma_i = \sigma^* \setminus \sigma_i',$$
$$P_i = \{s \in \sigma_i \mid s \in \rho_i \vee d(\sigma_i, s) \cap \rho_i \neq \varnothing \},$$

\hrulefill


Введём обозначения: $gen_i(s) = gen_{d(\sigma_i, s)}(s)$, $gen^*(s) = gen_{d(\sigma^*, s)}(s)$.\\
Рассмотрим различные случаи для $s \in \sigma_i'$ и действия, которые необходимо произвести с набором порождений $gen^d(\sigma_i)$, чтобы получить $gen^d(\sigma^*)$: \\

\begin{tabular}{ | p{3.5cm} || p{3.5cm} | p{3.5cm} | p{3.5cm} |}
	\hline
	Действие & $d(\sigma_i, s) \cap \rho_i \neq \varnothing$ & $d(\sigma_i, s) \cap \rho_i = \varnothing$ & $d(\sigma_i, s)$ не опр. \\ \hline
	& & & \\[-1.1em] \hline
	$d(\sigma^*, s) \cap \alpha_i \neq \varnothing$ & вычесть $gen_i(s)$, добавить $gen^*(s)$ & не бывает по предположению & добавить $gen^*(s)$ \\ \hline
	$d(\sigma^*, s) \cap \alpha_i = \varnothing$ & не бывает по предположению & взять $gen_i(s)$ & добавить $gen^*(s)$ \\ \hline
	$d(\sigma^*, s)$ не опр. & вычесть $gen_i(s)$ & вычесть $gen_i(s)$ & ничего не делать \\
	\hline
\end{tabular} \\\\

Введём ещё $Q_i = \{s \mid d(\sigma_i, s) \cap \rho_i = \varnothing$ \& $d(\sigma^*, s) \cap \alpha_i = \varnothing\}$ (центральная ячейка). На самом деле (см. таблицу) $Q_i = \sigma_i \setminus P_i \setminus \{s \mid d(\sigma^*, s) \text{ не опр.}\} \setminus \{s \mid d(\sigma_i, s) \text{ не опр.}\}$.

Тогда рассмотрим произвольное $i$ и произвольное $s \in \sigma^*$. Либо $s \in \alpha_i$, либо $s \in \sigma_i'$. Во втором случае либо $s \in Q_i$, и тогда по допущению (b) $gen^*(s) =  gen_i(s)$; либо нет, и тогда мы не можем использовать $gen_i(s)$.

Теперь рассмотрим произвольное $s \in \sigma^*$. Либо $\exists i_s$ : $s \in Q_i$, либо $\forall i \in [1:n]$ : $s \in \alpha_i \cup (\sigma_i' \setminus Q_i) = \alpha_i \cup \{s | d(\sigma_i, s) \cap \rho_i \neq \varnothing \vee d(\sigma_i, s) \text{ не опр.} \vee d(\sigma^*, s) \cap \alpha_i \neq \varnothing \vee d(\sigma^*, s) \text{ не опр.} \}$.

Во втором случае~--- если принадлежит всем, то принадлежит и пересечению:
$$s \in \bigcap\limits_{i = 1}^n \alpha_i \cup (\sigma_i' \setminus Q_i),$$
и тогда 
$$gen^*(s) \subseteq \bigcup\limits_{t \in \bigcap\limits_{i = 1}^n \alpha_i \cup (\sigma_i' \setminus Q_i)} gen^*(t).$$
Для $s$ таких, что $d(\sigma^*, s)$ не определено, $gen^*(s) = \varnothing$ независимо от $i$, поэтому можно исключить их из рассмотрения. Тогда получается, что пересекать достаточно только множества $A_i$, где
$$A_i = \alpha_i \cup \{s \in \sigma_i' \mid (d(\sigma_i, s) \cap \rho_i \neq \varnothing \;\&\; d(\sigma^*, s) \cap \alpha_i \neq \varnothing) \vee (d(\sigma_i, s) \text{ не опр.} \;\&\; d(\sigma^*, s) \text{ опред.})\}.$$

Но ничего не изменится (см. таблицу), если в качестве $A_i$ рассматривать
$$A_i = \alpha_i \cup \{s \in \sigma_i\setminus\rho_i \mid d(\sigma_i, s) \cap \rho_i \neq \varnothing \vee d(\sigma_i, s) \text{ не опр.}\}$$
или
$$A_i = \alpha_i \cup \{s \in \sigma_i\setminus\rho_i \mid d(\sigma^*, s) \cap \alpha_i \neq \varnothing \vee d(\sigma_i, s) \text{ не опр.}\}$$

Отсюда гипотеза:
$$gen^d(\sigma^*) = \left( \bigcup\limits_{i = 1}^n \bigcup\limits_{s \in Q_i} gen_{d(\sigma_i, s)}(s) \right) \cup \bigcup\limits_{s \in \bigcap\limits_{i = 1}^n A_i} gen_{d(\sigma^*, s)}(s)$$

Идея доказательства, неформально: в первом члене объединения мы перечисляем те входы, для которых можно переиспользовать порождение; те же, для которых нет ни одного порождения, которое можно было бы переиспользовать, попадают во второй член объединения. Таким образом, все порождения из левой части равенства присутствуют так или иначе в правой (либо как переиспользованные, либо как сгенерированные для нового контекста), в то же время в правой части не появится лишних порождений по построению: всё, что генерируется, генерируется по необходимости.

\end{comment}
\end{document}