%\documentclass[a4paper,12pt]{report} %%%{article}

\usepackage{cmap} % searchable PDFs
\usepackage[T2A]{fontenc} % scalable fonts
\usepackage[utf8]{inputenc} % input in UTF8
\usepackage[english,russian]{babel} % dashes on linebreaks
\usepackage{indentfirst} % indents in paragraphs
\usepackage{amstext,amssymb,amsfonts,amsmath,mathtext,enumerate,float}
\usepackage[left=25mm,right=2cm,top=2cm,bottom=2cm,bindingoffset=0cm]{geometry}
\usepackage[unicode]{hyperref}
\usepackage{graphicx}
\usepackage{ulem} % strikethrough
\usepackage{verbatim} % multiline comments
\usepackage{hhline}

\usepackage{listings}
\usepackage{color}
 
\lstset{extendedchars=\true}
  
\definecolor{dkgreen}{rgb}{0,0.6,0}
\definecolor{gray}{rgb}{0.5,0.5,0.5}
\definecolor{mauve}{rgb}{0.58,0,0.82}
 
\lstset{ %
  columns=flexible,
%  language=C,                     % the language of the code
  basicstyle=\footnotesize\ttfamily,       % the size of the fonts that are used for the code
  numbers=left,                   % where to put the line-numbers
  numberstyle=\tiny\color{gray},  % the style that is used for the line-numbers
  stepnumber=1,                   % the step between two line-numbers. If it's 1, each line will be numbered
  numbersep=5pt,                  % how far the line-numbers are from the code
  backgroundcolor=\color{white},  % choose the background color. You must add \usepackage{color}
  showspaces=false,               % show spaces adding particular underscores
  showstringspaces=false,         % underline spaces within strings
  showtabs=false,                 % show tabs within strings adding particular underscores
%  frame=single,                   % adds a frame around the code
  rulecolor=\color{black},        % if not set, the frame-color may be changed on line-breaks within not-black text (e.g. comments (green here))
  tabsize=4,                      % sets default tabsize
  captionpos=b,                   % sets the caption-position to bottom
  breaklines=true,                % sets automatic line breaking
  breakatwhitespace=false,        % sets if automatic breaks should only happen at whitespace
  title=\lstname,                 % show the filename of files included with \lstinputlisting; also try caption instead of title
  keywordstyle=\color{blue},      % keyword style
  commentstyle=\color{dkgreen},   % comment style
  stringstyle=\color{mauve},      % string literal style
%  escapeinside={\%*}{*)},         % if you want to add LaTeX within your code
  mathescape=true,
  morekeywords={*,...},           % if you want to add more keywords to the set
  deletekeywords={...},           % if you want to delete keywords from the given language
  aboveskip=1em,
  belowskip=-1em
}


\usepackage{mathtools}
\newcommand{\defeq}{\vcentcolon=}
\newcommand{\eqdef}{=\vcentcolon}

\renewcommand{\contentsname}{Содержание} 
\setcounter{secnumdepth}{0}
\setcounter{tocdepth}{3}

% \linespread{1.3}
\sloppy % align=justify


%\begin{document}

\section{Введение}
Сборка проекта в интегрированных средах разработки (integrated development environment, IDE)~--- это процесс генерации низкоуровневых артефактов из исходных файлов высокого уровня (программного кода, ресурсов и т.п.) по правилам, описанным в проекте.

Кэш компиляции~--- это некоторая структура данных, которая отражает мгновенное состояние проекта с точки зрения его реализации на низком уровне (например, для языка Java это совокупность класс-файлов плюс информация о зависимостях, соответствиях класс-файлов исходным файлам и т.д.). Такое хранилище может быть использовано (и используется в современных средах разработки) в процессе инкрементальной компиляции.

Инкрементальная компиляция представляет собой следующую оптимизацию процесса сборки проекта: вместо полной компиляции всего проекта с нуля перекомпилируются только изменённые со времени последней компиляции файлы, а также зависимые от них, на которые повлияли изменения (а также зависимые от них, и так далее~--- можно представить это как обход графа зависимостей в ширину). Например, если в одном из исходных Java-файлов разработчик изменил сигнатуру некоторого метода, то нужно перекомпилировать не только этот файл, но и все файлы, где присутствует вызов этого метода (т.е. зависимые). В целях экономии вычислительных ресурсов информация о зависимостях подобного рода сохраняется в кэше компиляции, чтобы при следующем запуске сборки проекта можно было эффективно вычислять набор файлов, которые подлежат перекомпиляции. В этот же кэш попадают и класс-файлы~--- в случае, когда ни исходный файл, ни те файлы, от которых он зависит, не изменились со времени последней компиляции, ясно, что можно переиспользовать результат его последней компиляции~--- взять сохранённый класс-файл, вместо того чтобы тратить вычислительные ресурсы на компиляцию.

Кэш компиляции подвергается обновлению при каждой компиляции, чтобы находиться в актуальном состоянии, пригодном к использованию при следующем сеансе компиляции. Если над проектом работает несколько человек, у каждого из которых есть локальная рабочая копия репозитория, то у каждого нарастает свой локальный кэш компиляции, при этом б\'{о}льшая его часть у всех одинакова. В процессе такой коллективной разработки случаются ситуации, когда состояние проекта в локальной копии разработчика существенно отличается от состояния, для которого был посчитан его локальный кэш компиляции. Тогда инкрементальная компиляция теряет свои преимущества и мало чем отличается от процедуры полной сборки проекта с нуля. Примерами таких случаев могут служить cold start~--- сборка проекта в ситуации отсутствия кэша вообще~--- или переключение между ветками (``branches'')~--- например, основной и релизной ветками. Однако можно воспользоваться тем, что разработчиков всё-таки несколько и предложить способ переиспользования кэшей одних разработчиков другими, чтобы в таких ситуациях не приходилось каждый раз строить их с нуля.

В данной работе предлагается такой способ переиспользования кэшей. Рассматривается главным образом процесс компиляции исходных файлов на языке Java, однако полученные результаты достаточно абстрактны, чтобы подход можно было применить к другим языкам программирования. Заметим, что подход применим не только к компиляции, но также и к другим вычислительно сложным процессам, которые получают на вход некоторый набор исходных файлов и выдают в качестве результата некоторый набор порождений (примером такого процесса может служить построение индексов).

%\end{document}