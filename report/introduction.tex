\chapter{Введение}
Кэш компиляции --- это некоторая структура данных, которая отражает мгновенное состояние проекта с точки зрения его реализации на низком уровне (например, совокупность класс-файлов плюс информация о зависимостях и т.д.) Если над проектом работает несколько человек, то у каждого нарастает свой кэш компиляции, при этом большая его часть у всех одинакова. Предлагается подумать над техникой разделения таких кэшей, чтобы не нужно было каждый раз строить их от печки.

Вопросы для исследования: Когда эффективно кэширование? Когда выгоднее компилировать с нуля? Что быстрее - передача по сети или локальная перекомпиляция? Нужны эксперименты: разные операционные системы, разные файловые системы, разные размеры проектов, размеры данных, пересылаемых по сети.

Сейчас: Локальное кеширование, изменение файлов отслеживается по timestamp. Хранятся абсолютные пути к файлам, при перемещении проекта в другую директорию кэш теряется. Инкрементальная компиляция: перекомпилируются изменённые файлы и те зависимые от них, на которые повлияли изменения (а также зависимые от них, и т.д. - обход в ширину). Например, если происходит изменение public static final поля - нужно перекомпилировать все файлы, где есть использование такого поля (т.к. значение подставляется при компиляции).

Актуально решать задачу для случаев: 1) cold start - полный билд проекта; 2) переключение между бранчами - напр., основная и релизная ветка.