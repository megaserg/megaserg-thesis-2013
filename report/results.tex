\section{Заключение}

В рамках дипломной работы были получены следующие результаты:

\begin{itemize}

	\item Построена формальная модель, описывающая предметную область.

	\item Построена аксиоматика, отражающая свойства и ограничения реального мира и позволяющая доказывать нетривиальные утверждения.

	\item В терминах построенной формальной модели описана задача инкрементальной компиляции, сформулирована и доказана соответствующая теорема.

	\item Сформулирована и доказана теорема о переиспользовании порождений.

\end{itemize}

В качестве возможного продолжения работы планируется реализовать прототип, демонстрирующий работоспособность подхода на базе доказанных теорем. В качестве основы для реализации прототипа предполагается использовать среду разработки IntelliJ IDEA\footnote{\url{http://www.jetbrains.com/idea/}}, разрабатываемой компанией JetBrains. Чтобы проверить работоспособность и эффективность предложенного подхода, предлагается провести на реализованном прототипе эксперименты, в ходе которых дать ответы на следующие вопросы: В каких случаях эффективно кэширование? В каких выгоднее компилировать с нуля? Что быстрее~--- передача кэшированных класс-файлов по сети или локальная перекомпиляция? В ходе экспериментов требуется рассмотреть различные операционные системы, различные файловые системы, различные дисковые носители (HDD vs. SSD), а также варьировать размеры проектов и размеры данных, пересылаемых по сети.

Полученные результаты являются достаточно общими для того, чтобы подход можно было применить и к другим языкам программирования. Заметим, что подход применим не только к компиляции, но также и к другим вычислительно сложным процессам, которые получают на вход некоторый набор исходных файлов и выдают в качестве результата некоторый набор порождений. Примером может служить задача переиспользования индексов, используемых средами разработки для реализации интеллектуальных функций вроде ``Find Usages'', ``Find Implementations'', рефакторинга и т.п. Построение таких индексов с нуля для крупных проектов занимает достаточно много времени, поскольку требует обхода и чтения всех исходных файлов проекта. Используя идеи, сходные с упомянутыми для задачи переиспользования порождений, можно добиться существенного сокращения времени построения этих индексов.