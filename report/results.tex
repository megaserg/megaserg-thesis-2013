\section*{Заключение}
В качестве возможного продолжения работы планируется реализовать прототип, демонстрирующий работоспособность подхода на базе доказанных теорем. В качестве основы для реализации прототипа предполагается использовать среду разработки IntelliJ IDEA, разрабатываемой компанией JetBrains. Чтобы проверить работоспособность и эффективность предложенного подхода, предлагается провести на реализованном прототипе эксперименты, в ходе которых дать ответы на следующие вопросы: в каких случаях эффективно кэширование? в каких выгоднее компилировать с нуля? что быстрее~--- передача кэшированных класс-файлов по сети или локальная перекомпиляция? В ходе экспериментов требуется рассмотреть различные операционные системы, различные файловые системы, различные дисковые носители (HDD и SSD), а также варьировать размеры проектов и размеры данных, пересылаемых по сети.

Полученные результаты являются достаточно общими для того, чтобы подход можно было применить и к другим языкам программирования. Заметим, что описанный подход применим не только к компиляции, но также и к другим вычислительно сложным процессам, которые получают на вход некоторый набор исходных файлов и выдают в качестве результата некоторый набор порождений. Примером может служить задача переиспользования индексов, используемых средами разработки для реализации интеллектуальных функций вроде ``Find Usages'', ``Find Implementations'', рефакторинга и т.п. Построение таких индексов с нуля для крупных проектов занимает достаточно много времени, поскольку требует обхода и чтения всех исходных файлов проекта. Используя идеи, сходные с упомянутыми для задачи переиспользования порождений, можно добиться существенного сокращения времени построения этих индексов.

Автор выражает благодарность доценту кафедры системного программирования математико-механического факультета СПбГУ, \mbox{к.ф.-м.н.} Д.~Ю.~Булычеву за научное руководство и разработчику \mbox{ООО~``ИнтеллиДжей Лабс''}, \mbox{к.ф.-м.н.} Н.~В.~Чашникову за постановку задачи и полезные консультации.