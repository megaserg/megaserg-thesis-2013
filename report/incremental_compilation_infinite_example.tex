\documentclass[a4paper,12pt]{report} %%%{article}

\usepackage{cmap} % searchable PDFs
\usepackage[T2A]{fontenc} % scalable fonts
\usepackage[utf8]{inputenc} % input in UTF8
%\usepackage[english,russian]{babel} % dashes on linebreaks
\usepackage{indentfirst} % indents in paragraphs
\usepackage{amstext,amssymb,amsfonts,amsmath,mathtext,enumerate,float}
\usepackage[left=25mm,right=2cm,top=2cm,bottom=2cm,bindingoffset=0cm]{geometry}
\usepackage[unicode]{hyperref}
\usepackage{graphicx}
\usepackage{ulem} % strikethrough
\usepackage{verbatim} % multiline comments
\usepackage{hhline}

\usepackage{listings}
\usepackage{color}
 
\definecolor{dkgreen}{rgb}{0,0.6,0}
\definecolor{gray}{rgb}{0.5,0.5,0.5}
\definecolor{mauve}{rgb}{0.58,0,0.82}
 
\lstset{ %
  columns=flexible,
%  language=C,                     % the language of the code
  basicstyle=\footnotesize\ttfamily,       % the size of the fonts that are used for the code
  numbers=left,                   % where to put the line-numbers
  numberstyle=\tiny\color{gray},  % the style that is used for the line-numbers
  stepnumber=1,                   % the step between two line-numbers. If it's 1, each line 
                                  % will be numbered
  numbersep=5pt,                  % how far the line-numbers are from the code
  backgroundcolor=\color{white},  % choose the background color. You must add \usepackage{color}
  showspaces=false,               % show spaces adding particular underscores
  showstringspaces=false,         % underline spaces within strings
  showtabs=false,                 % show tabs within strings adding particular underscores
%  frame=single,                   % adds a frame around the code
  rulecolor=\color{black},        % if not set, the frame-color may be changed on line-breaks within not-black text (e.g. comments (green here))
  tabsize=4,                      % sets default tabsize
  captionpos=b,                   % sets the caption-position to bottom
  breaklines=true,                % sets automatic line breaking
  breakatwhitespace=false,        % sets if automatic breaks should only happen at whitespace
  title=\lstname,                 % show the filename of files included with \lstinputlisting;
                                  % also try caption instead of title
  keywordstyle=\color{blue},      % keyword style
  commentstyle=\color{dkgreen},   % comment style
  stringstyle=\color{mauve},      % string literal style
%  escapeinside={\%*}{*)},         % if you want to add LaTeX within your code
  mathescape=true,
  morekeywords={*,...},           % if you want to add more keywords to the set
  deletekeywords={...}            % if you want to delete keywords from the given language
}

\lstset{
literate={а}{{\selectfont\char224}}1
{б}{{\selectfont\char225}}1
{в}{{\selectfont\char226}}1
{г}{{\selectfont\char227}}1
{д}{{\selectfont\char228}}1
{е}{{\selectfont\char229}}1
{ё}{{\"e}}1
{ж}{{\selectfont\char230}}1
{з}{{\selectfont\char231}}1
{и}{{\selectfont\char232}}1
{й}{{\selectfont\char233}}1
{к}{{\selectfont\char234}}1
{л}{{\selectfont\char235}}1
{м}{{\selectfont\char236}}1
{н}{{\selectfont\char237}}1
{о}{{\selectfont\char238}}1
{п}{{\selectfont\char239}}1
{р}{{\selectfont\char240}}1
{с}{{\selectfont\char241}}1
{т}{{\selectfont\char242}}1
{у}{{\selectfont\char243}}1
{ф}{{\selectfont\char244}}1
{х}{{\selectfont\char245}}1
{ц}{{\selectfont\char246}}1
{ч}{{\selectfont\char247}}1
{ш}{{\selectfont\char248}}1
{щ}{{\selectfont\char249}}1
{ъ}{{\selectfont\char250}}1
{ы}{{\selectfont\char251}}1
{ь}{{\selectfont\char252}}1
{э}{{\selectfont\char253}}1
{ю}{{\selectfont\char254}}1
{я}{{\selectfont\char255}}1
{А}{{\selectfont\char192}}1
{Б}{{\selectfont\char193}}1
{В}{{\selectfont\char194}}1
{Г}{{\selectfont\char195}}1
{Д}{{\selectfont\char196}}1
{Е}{{\selectfont\char197}}1
{Ё}{{\"E}}1
{Ж}{{\selectfont\char198}}1
{З}{{\selectfont\char199}}1
{И}{{\selectfont\char200}}1
{Й}{{\selectfont\char201}}1
{К}{{\selectfont\char202}}1
{Л}{{\selectfont\char203}}1
{М}{{\selectfont\char204}}1
{Н}{{\selectfont\char205}}1
{О}{{\selectfont\char206}}1
{П}{{\selectfont\char207}}1
{Р}{{\selectfont\char208}}1
{С}{{\selectfont\char209}}1
{Т}{{\selectfont\char210}}1
{У}{{\selectfont\char211}}1
{Ф}{{\selectfont\char212}}1
{Х}{{\selectfont\char213}}1
{Ц}{{\selectfont\char214}}1
{Ч}{{\selectfont\char215}}1
{Ш}{{\selectfont\char216}}1
{Щ}{{\selectfont\char217}}1
{Ъ}{{\selectfont\char218}}1
{Ы}{{\selectfont\char219}}1
{Ь}{{\selectfont\char220}}1
{Э}{{\selectfont\char221}}1
{Ю}{{\selectfont\char222}}1
{Я}{{\selectfont\char223}}1
}
\usepackage{mathtools}
\newcommand{\defeq}{\vcentcolon=}
\newcommand{\eqdef}{=\vcentcolon}

\renewcommand{\contentsname}{Содержание} 
\setcounter{secnumdepth}{0}
\setcounter{tocdepth}{3}

%\sloppy % align=justify

\begin{document}

\section{Основные определения}

$\Sigma$ --- множество входов, $\Omega$ --- множество выходов (н.б.ч.с.). Функция порождения
выходов по входам (частичная):

$$
gen : 2^\Omega\to 2^\Sigma\to 2^\Omega
$$

Аксиомы функции порождения:

\begin{itemize}
	\item если $gen(\omega,\sigma)$ --- определена, то $\omega\cap gen(\omega,\sigma)=\varnothing$;
	\item если $gen(\omega,\sigma)=\omega^\prime$, то существует единственное дизъюнктное разбиение $\omega^\prime=\bigcup^\varnothing_{s\in\sigma}\omega^\prime_s$, 
	удовлетворяющее свойству 

	$$\forall s\in\sigma : gen(\omega\cup\omega^\prime\setminus\omega^\prime_s,\{s\})=\omega^\prime_s$$

	\item $\forall s\in\Sigma,\; \forall\omega,\:\omega^\prime\subseteq\Omega:$ если $gen(\omega,\{s\})$ и $gen(\omega^\prime,\{s\})$ ---
	определены, то $gen(\omega,\{s\})=gen(\omega^\prime,\{s\})$;

	\item если $gen(\omega,\{s\})$ --- определена, то в $\omega$ существует наименьшее по включению подмножество $d_\omega(s)$, такое, что
	$gen(d_\omega(s), \{s\})$ --- определена;

	\item если $gen(\omega,\sigma)=\omega^\prime$ и для какого-то $s\in\sigma$ существует $s_1\in\sigma$, такой, что
	% $d_\omega(s)\cap \omega^\prime_{s_1}\ne\varnothing$, 
	$d_{\omega\cup\omega^\prime\setminus\omega^\prime_s}(s) \cap \omega^\prime_{s_1}\ne\varnothing$, 
	то $gen(\omega,\sigma\setminus s_1)$ --- не определена;

	\item если $gen(\omega,\sigma_1)\cap gen(\omega,\sigma_2)\ne\varnothing$, то $gen(\omega,\sigma_1\cup\sigma_2)$ --- не определена.
\end{itemize}

\hrulefill

О шестой аксиоме: похоже, она выводится из остальных:

Пусть $gen(\omega,\sigma_1) = \omega_1$, $gen(\omega,\sigma_2) = \omega_2$, $\omega_1\cap\omega_2\ne\varnothing$. Предположим, что $gen(\omega,\sigma_1\cup\sigma_2)$ определена и равна $\omega^\prime$. Тогда по аксиоме о дизъюнктном разбиении можно (объединяя порождения для соответствующих $s$ и учитывая, что $\sigma_1\cap\sigma_2 = \varnothing$) разделить $\omega^\prime$ на $\omega^\prime_1$ и $\omega^\prime_2$, где $\omega^\prime_1 = gen(\omega\cup\omega^\prime_2,\sigma_1)$, $\omega^\prime_2 = gen(\omega\cup\omega^\prime_1,\sigma_2)$, $\omega^\prime_1 \cap \omega^\prime_2 = \varnothing$. Но по аксиоме ``определены-значит-равны'' $\omega_1 = \omega^\prime_1$ и $\omega_2 = \omega^\prime_2$; таким образом, $\omega_1$ и $\omega_2$ одновременно пересекаются и не пересекаются, значит, предположение об определённости $gen(\omega,\sigma_1\cup\sigma_2)$ неверно.
\\

Наконец, очевидное следствие из аксиомы о дизъюнктном разбиении:

\begin{itemize}
	\item если $gen(\omega,\sigma)=\omega^\prime$, то для любого $\tau \subseteq \sigma$ можно определить $\omega^\prime_\tau =\bigcup\limits_{s\in\tau}\omega^\prime_s$, при этом $\tau_1 \cap \tau_2 = \varnothing \Leftrightarrow \omega^\prime_{\tau_1} \cap \omega^\prime_{\tau_2} = \varnothing$.
\end{itemize}

Если для множества входов $\sigma$ в каком-то контексте определено $gen(..., \sigma)$, то по третьей аксиоме значения $gen(..., \sigma)$ для всех контекстов, где такое значение определено, равны; будем говорить, что они равны $\omega_\sigma$ (без верхнего индекса).

\newpage

Построим пример. $\mathbb{V} = \{a, b\}$, $\Omega = \mathbb{V} \times \mathbb{N}_0
$, $\Sigma = \{f_a, f_b\}$, где $f_v : 2^\Omega \to \Omega$ и
$$f_v(\omega) = \begin{cases}
    (v, 1 + \max(\min\limits_{(a, n)\in\omega} n, \min\limits_{(b, n)\in\omega} n)), & \mbox{если } \exists n: (v, n) \in \omega\\
	(v, 0), & \mbox{если } \nexists n: (v, n) \in \omega
\end{cases}$$
Функция $gen : 2^\Omega \times 2^\Sigma \to 2^\Omega$ определена как $gen(\omega, \{f_{v_i}\}_i) = \bigcup\limits_i f_{v_i}(\omega) = \bigcup\limits_i \{(v_i, n_{v_i})\}$. Контекст можно понимать как набор пар ``переменная-значение''. При этом потребуем, что если для какого-то $i$ $(v_i, n_{v_i}) \in \omega$, то $gen$ не определено.

Заметим, что в $\omega$ могут входить несколько пар с одной и той же переменной на первом месте. При этом для $f_v(\omega) = (v, n)$ верно, что $n$ больше минимума значений переменной $v$ в контексте.

Определим $d_\omega(f_v)$ как $\varnothing$, если в $\omega$ нет пар с $v$; в противном случае для каждой переменной, с которой есть пары в $\omega$, включим в $d_\omega(f_v)$ ту пару, где она содержится с минимальным значением.\\

Проверим, удовлетворяет ли построенный пример аксиомам.
\begin{itemize}
\item Первая выполняется, потому что $gen$ по построению не определено в ситуации пересечения порождений и контекста.
\item Вторая выполняется: действительно, в результате $gen$ каждой функции-входу соответствует ровно одна пара ``переменная-значение''; при этом, если в $\sigma$ более одной функции (то есть две), то можно оставить во входе одну из них и перенести результат другой в контекст; поскольку этот результат больше минимума присутствующих в контексте значений, то результат вычисления первой функции в новом контексте не изменится (то есть получим равенство $gen(\omega\cup\omega^\prime\setminus\omega^\prime_s,\{s\}) = \omega^\prime_s$). 
\item Четвёртая аксиома выполняется: действительно, для вычисления $gen$ необходимым и достаточным является то подмножество контекста, которое мы обозначили $d_\omega(f_v)$; при этом результат вычислений не изменяется при удалении из контекста пар, не принадлежащих $d_\omega(f_v)$, что удовлетворяет седьмой аксиоме.
\item Описанная в пятой аксиоме ситуация никогда не возникает: действительно, $d_{\omega\cup\omega^\prime\setminus\omega^\prime_s}(s)$ на самом деле равно $d_{\omega}(s)$ (потому что в $\omega\cup\omega^\prime\setminus\omega^\prime_s$ минимумы значений совпадают с минимумами в $\omega$), и не может пересекаться с порождениями $s_1$.
\item Ситуация из шестого утверждения тоже никогда не возникает, потому что из дизъюнктности входов следует дизъюнктность порождений.
\end{itemize}

Теперь рассмотрим такую последовательность компиляций:
$$gen(\varnothing, \{f_a, f_b\}) = \{(a,0), (b,0)\} = \omega^{0,0}$$
$$gen(\omega^{0,0}, \{f_a\}) = \{(a,1)\}; \omega^{1,0} = \omega^{0,0} \setminus \omega_{0,a} \cup \omega_{1,a} = \{(a,1), (b,0)\}$$
$$gen(\omega^{1,0}, \{f_b\}) = \{(b,2)\}; \omega^{1,1} = \omega^{1,0} \setminus \omega_{0,b} \cup \omega_{1,b} = \{(a,1), (b,2)\}$$
$$gen(\omega^{1,1}, \{f_a\}) = \{(a,3)\}; \omega^{2,1} = \omega^{1,1} \setminus \omega_{1,a} \cup \omega_{2,a} = \{(a,3), (b,2)\}$$
$$gen(\omega^{2,1}, \{f_b\}) = \{(b,4)\}; \omega^{2,2} = \omega^{2,1} \setminus \omega_{1,b} \cup \omega_{2,b} = \{(a,3), (b,4)\}$$
$$...$$

Получаем бесконечную последовательность компиляций, которая не сходится.

\end{document}